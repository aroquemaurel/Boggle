\documentclass{book}
\usepackage[a4paper,top=2.5cm,bottom=2.5cm,left=2.5cm,right=2.5cm]{geometry}
\usepackage{makeidx}
\usepackage{natbib}
\usepackage{graphicx}
\usepackage{multicol}
\usepackage{float}
\usepackage{listings}
\usepackage{color}
\usepackage{ifthen}
\usepackage[table]{xcolor}
\usepackage{textcomp}
\usepackage{alltt}
\usepackage{ifpdf}
\ifpdf
\usepackage[pdftex,
            pagebackref=true,
            colorlinks=true,
            linkcolor=blue,
            unicode
           ]{hyperref}
\else
\usepackage[ps2pdf,
            pagebackref=true,
            colorlinks=true,
            linkcolor=blue,
            unicode
           ]{hyperref}
\usepackage{pspicture}
\fi
\usepackage[utf8]{inputenc}
\usepackage[french]{babel}

\usepackage{mathptmx}
\usepackage[scaled=.90]{helvet}
\usepackage{courier}
\usepackage{sectsty}
\usepackage{amssymb}
\usepackage[titles]{tocloft}
\usepackage{doxygen}
\lstset{language=C++,inputencoding=utf8,basicstyle=\footnotesize,breaklines=true,breakatwhitespace=true,tabsize=4,numbers=left }
\makeindex
\setcounter{tocdepth}{3}
\renewcommand{\footrulewidth}{0.4pt}
\renewcommand{\familydefault}{\sfdefault}
\hfuzz=15pt
\setlength{\emergencystretch}{15pt}
\hbadness=750
\tolerance=750
\begin{document}
\hypersetup{pageanchor=false,citecolor=blue}
\begin{titlepage}
\vspace*{7cm}
\begin{center}
{\Large Jeu de Boggle \\[1ex]\large 1 }\\
\vspace*{1cm}
{\large Généré par Doxygen 1.8.1.2}\\
\vspace*{0.5cm}
{\small Mardi Avril 30 2013 10:07:23}\\
\end{center}
\end{titlepage}
\clearemptydoublepage
\pagenumbering{roman}
\tableofcontents
\clearemptydoublepage
\pagenumbering{arabic}
\hypersetup{pageanchor=true,citecolor=blue}
\chapter{Index des structures de données}
\section{Structures de données}
Liste des structures de données avec une brève description :\begin{DoxyCompactList}
\item\contentsline{section}{\hyperlink{structCase}{Case} }{\pageref{structCase}}{}
\item\contentsline{section}{\hyperlink{structDico}{Dico} }{\pageref{structDico}}{}
\item\contentsline{section}{\hyperlink{structJeu}{Jeu} }{\pageref{structJeu}}{}
\item\contentsline{section}{\hyperlink{structPlateau}{Plateau} }{\pageref{structPlateau}}{}
\item\contentsline{section}{\hyperlink{structSolution}{Solution} }{\pageref{structSolution}}{}
\end{DoxyCompactList}

\chapter{Index des fichiers}
\section{Liste des fichiers}
Liste de tous les fichiers documentés avec une brève description \-:\begin{DoxyCompactList}
\item\contentsline{section}{\hyperlink{couple_8h}{couple.\-h} \\*Structure de données d'un couple Structure de données d'un couple de valeurs entières }{\pageref{couple_8h}}{}
\item\contentsline{section}{\hyperlink{dictionnaire_8c}{dictionnaire.\-c} \\*Gestion du dictionnaire Fonctions de gestion du dictionnaire, le dictionnaire doit être dans un fichier texte avec un mot différent par ligne }{\pageref{dictionnaire_8c}}{}
\item\contentsline{section}{\hyperlink{dictionnaire_8h}{dictionnaire.\-h} \\*Gestion du dictionnaire Fonctions de gestion du dictionnaire, le dictionnaire doit être dans un fichier texte avec un mot différent par ligne }{\pageref{dictionnaire_8h}}{}
\item\contentsline{section}{\hyperlink{interfaceNcurses_8c}{interface\-Ncurses.\-c} \\*Affichage avec la bibliothèque Ncurses Affiche et permet de jouer au jeu de Boggle avec un affichage utilisant la bibliothèque Ncurses }{\pageref{interfaceNcurses_8c}}{}
\item\contentsline{section}{\hyperlink{interfaceNcurses_8h}{interface\-Ncurses.\-h} \\*Affichage avec la bibliothèque Ncurses Affiche et permet de jouer au jeu de Boggle avec un affichage utilisant la bibliothèque Ncurses }{\pageref{interfaceNcurses_8h}}{}
\item\contentsline{section}{\hyperlink{interfaceTexte_8c}{interface\-Texte.\-c} \\*Tous les affichages en mode texte seulement Permet l'affichage du jeu en mode texte uniquement, les saisies se font au clavier }{\pageref{interfaceTexte_8c}}{}
\item\contentsline{section}{\hyperlink{interfaceTexte_8h}{interface\-Texte.\-h} \\*Tous les affichages en mode texte seulement Permet l'affichage du jeu en mode texte uniquement, les saisies se font au clavier }{\pageref{interfaceTexte_8h}}{}
\item\contentsline{section}{\hyperlink{jeu_8c}{jeu.\-c} \\*Gestion du jeu de Boggle Les fonctions permettant de gérer le \hyperlink{structJeu}{Jeu} de Boggle }{\pageref{jeu_8c}}{}
\item\contentsline{section}{\hyperlink{jeu_8h}{jeu.\-h} \\*Gestion du jeu de Boggle Les fonctions permettant de gérer le \hyperlink{structJeu}{Jeu} de Boggle }{\pageref{jeu_8h}}{}
\item\contentsline{section}{\hyperlink{main_8c}{main.\-c} }{\pageref{main_8c}}{}
\item\contentsline{section}{\hyperlink{plateau_8c}{plateau.\-c} \\*Gestion du plateau de \hyperlink{structJeu}{Jeu} Fonctions se rapportant à un plateau de jeu, sera sert à gérer la grille de Boggle }{\pageref{plateau_8c}}{}
\item\contentsline{section}{\hyperlink{plateau_8h}{plateau.\-h} \\*Gestion du plateau de \hyperlink{structJeu}{Jeu} Fonctions se rapportant à un plateau de jeu, sera sert à gérer la grille de Boggle }{\pageref{plateau_8h}}{}
\item\contentsline{section}{\hyperlink{resolveur_8c}{resolveur.\-c} \\*Gestion du résolveur Permet de résoudre une grille de Boggle, une fois résolue, la solution se trouve dans le module \hyperlink{structSolution}{Solution} }{\pageref{resolveur_8c}}{}
\item\contentsline{section}{\hyperlink{resolveur_8h}{resolveur.\-h} \\*Gestion du résolveur Permet de résoudre une grille de Boggle, une fois résolue, la solution se trouve dans le module \hyperlink{structSolution}{Solution} }{\pageref{resolveur_8h}}{}
\item\contentsline{section}{\hyperlink{solution_8c}{solution.\-c} \\*Gère la solution d'une grille de boggle Toutes les fonctions se rapportant à la solution d'une grille }{\pageref{solution_8c}}{}
\item\contentsline{section}{\hyperlink{solution_8h}{solution.\-h} \\*Gère la solution d'une grille de boggle Toutes les fonctions se rapportant à la solution d'une grille }{\pageref{solution_8h}}{}
\item\contentsline{section}{\hyperlink{util_8c}{util.\-c} \\*Fonctions utiles à l'ensemble du projet Toutes les fonctions de bases utiles au projet, ces fonctions travaillent sur des types de bae et ne sont pas spécfiques au projet, ce module permet de meux organiser le code }{\pageref{util_8c}}{}
\item\contentsline{section}{\hyperlink{util_8h}{util.\-h} \\*Fonctions utiles à l'ensemble du projet Toutes les fonctions de bases utiles au projet, ces fonctions travaillent sur des types de bae et ne sont pas spécfiques au projet, ce module permet de meux organiser le code }{\pageref{util_8h}}{}
\end{DoxyCompactList}

\chapter{Documentation des structures de données}
\hypertarget{structCase}{\section{Référence de la structure Case}
\label{structCase}\index{Case@{Case}}
}
\subsection*{Champs de données}
\begin{DoxyCompactItemize}
\item 
\hypertarget{structCase_aa153db46def2bd44f5e438a0f5695779}{unsigned char {\bfseries i}}\label{structCase_aa153db46def2bd44f5e438a0f5695779}

\item 
\hypertarget{structCase_aa29567a6d5b6753ecf7cff4567a77e1e}{unsigned char {\bfseries j}}\label{structCase_aa29567a6d5b6753ecf7cff4567a77e1e}

\end{DoxyCompactItemize}


La documentation de cette structure a été générée à partir du fichier suivant \-:\begin{DoxyCompactItemize}
\item 
case.\-h\end{DoxyCompactItemize}

\hypertarget{structDico}{
\section{Référence de la structure Dico}
\label{structDico}\index{Dico@{Dico}}
}


Dictionnaire Contient le dictionnaire du jeu.  




{\ttfamily \#include $<$dictionnaire.h$>$}

\subsection*{Champs de données}
\begin{DoxyCompactItemize}
\item 
FILE $\ast$ \hyperlink{structDico_aa608808df162aebf2b2bfbc4723cfad6}{dico}
\item 
long int \hyperlink{structDico_a1a8a33881f938f4ddd946edc97ae82f8}{marqueurs} \mbox{[}32\mbox{]}
\end{DoxyCompactItemize}


\subsection{Description détaillée}
Dictionnaire Contient le dictionnaire du jeu. 

\subsection{Documentation des champs}
\hypertarget{structDico_aa608808df162aebf2b2bfbc4723cfad6}{
\index{Dico@{Dico}!dico@{dico}}
\index{dico@{dico}!Dico@{Dico}}
\subsubsection[{dico}]{\setlength{\rightskip}{0pt plus 5cm}FILE$\ast$ {\bf Dico::dico}}}
\label{structDico_aa608808df162aebf2b2bfbc4723cfad6}
Fichier contenant le dictionnaire, celui-\/ci doit être un fichier texte avec un mot par ligne \hypertarget{structDico_a1a8a33881f938f4ddd946edc97ae82f8}{
\index{Dico@{Dico}!marqueurs@{marqueurs}}
\index{marqueurs@{marqueurs}!Dico@{Dico}}
\subsubsection[{marqueurs}]{\setlength{\rightskip}{0pt plus 5cm}long int {\bf Dico::marqueurs}\mbox{[}32\mbox{]}}}
\label{structDico_a1a8a33881f938f4ddd946edc97ae82f8}
Contient les positions de chacunes des lettres de l'alphabets dans le fichier 

La documentation de cette structure a été générée à partir du fichier suivant :\begin{DoxyCompactItemize}
\item 
dictionnaire.h\end{DoxyCompactItemize}

\hypertarget{structJeu}{\section{Référence de la structure Jeu}
\label{structJeu}\index{Jeu@{Jeu}}
}
\subsection*{Champs de données}
\begin{DoxyCompactItemize}
\item 
\hypertarget{structJeu_af2931c71f6de940ec7ed3fcd2f9d7158}{\hyperlink{structPlateau}{Plateau} {\bfseries plateau}}\label{structJeu_af2931c71f6de940ec7ed3fcd2f9d7158}

\item 
\hypertarget{structJeu_a6bda384c6fca4e83ae15ac69f2228a41}{\hyperlink{structDico}{Dico} {\bfseries dico}}\label{structJeu_a6bda384c6fca4e83ae15ac69f2228a41}

\item 
\hypertarget{structJeu_af48a5fddf1dc40b7c4a85918d73f201d}{time\-\_\-t {\bfseries timestamp\-Depart}}\label{structJeu_af48a5fddf1dc40b7c4a85918d73f201d}

\item 
\hypertarget{structJeu_ae8e1dca71285915404d1de4ff609a1d6}{\hyperlink{structSolution}{Solution} {\bfseries solution\-Utilisateur}}\label{structJeu_ae8e1dca71285915404d1de4ff609a1d6}

\end{DoxyCompactItemize}


La documentation de cette structure a été générée à partir du fichier suivant \-:\begin{DoxyCompactItemize}
\item 
jeu.\-h\end{DoxyCompactItemize}

\hypertarget{structPlateau}{\section{Référence de la structure Plateau}
\label{structPlateau}\index{Plateau@{Plateau}}
}


Grille du jeu Contient une grille permettant de jouer au Boggle.  




{\ttfamily \#include $<$plateau.\-h$>$}

\subsection*{Champs de données}
\begin{DoxyCompactItemize}
\item 
char $\ast$$\ast$ \hyperlink{structPlateau_a42411aa44b78d298681030ce3461f686}{grille}
\item 
\hyperlink{structCouple}{Couple} \hyperlink{structPlateau_af86602008c1cfce98187fd9b9b4e50ce}{taille\-Grille}
\item 
\hypertarget{structPlateau_aeb94d618db5af1b25a4f64042a4ad771}{int $\ast$$\ast$ {\bfseries grid}}\label{structPlateau_aeb94d618db5af1b25a4f64042a4ad771}

\item 
\hyperlink{structSolution}{Solution} \hyperlink{structPlateau_aed88af50891b17edcd3ba97d7098dfde}{solution}
\end{DoxyCompactItemize}


\subsection{Description détaillée}
Grille du jeu Contient une grille permettant de jouer au Boggle. 

\subsection{Documentation des champs}
\hypertarget{structPlateau_a42411aa44b78d298681030ce3461f686}{\index{Plateau@{Plateau}!grille@{grille}}
\index{grille@{grille}!Plateau@{Plateau}}
\subsubsection[{grille}]{\setlength{\rightskip}{0pt plus 5cm}char$\ast$$\ast$ Plateau\-::grille}}\label{structPlateau_a42411aa44b78d298681030ce3461f686}
La grille de boggle \hypertarget{structPlateau_aed88af50891b17edcd3ba97d7098dfde}{\index{Plateau@{Plateau}!solution@{solution}}
\index{solution@{solution}!Plateau@{Plateau}}
\subsubsection[{solution}]{\setlength{\rightskip}{0pt plus 5cm}{\bf Solution} Plateau\-::solution}}\label{structPlateau_aed88af50891b17edcd3ba97d7098dfde}
La solution complète de cette grille de Boggle \hypertarget{structPlateau_af86602008c1cfce98187fd9b9b4e50ce}{\index{Plateau@{Plateau}!taille\-Grille@{taille\-Grille}}
\index{taille\-Grille@{taille\-Grille}!Plateau@{Plateau}}
\subsubsection[{taille\-Grille}]{\setlength{\rightskip}{0pt plus 5cm}{\bf Couple} Plateau\-::taille\-Grille}}\label{structPlateau_af86602008c1cfce98187fd9b9b4e50ce}
La taille de la grille 

La documentation de cette structure a été générée à partir du fichier suivant \-:\begin{DoxyCompactItemize}
\item 
\hyperlink{plateau_8h}{plateau.\-h}\end{DoxyCompactItemize}

\hypertarget{structSolution}{\section{Référence de la structure Solution}
\label{structSolution}\index{Solution@{Solution}}
}
\subsection*{Champs de données}
\begin{DoxyCompactItemize}
\item 
\hypertarget{structSolution_ab511245410043c846351c6acf272d656}{char $\ast$$\ast$ {\bfseries mots}}\label{structSolution_ab511245410043c846351c6acf272d656}

\item 
\hypertarget{structSolution_a60e0881a49e593091d2210dd49f2f51a}{unsigned int {\bfseries nb\-Mots}}\label{structSolution_a60e0881a49e593091d2210dd49f2f51a}

\item 
\hypertarget{structSolution_a25fcaf3e9fa7e1a59a149f959f185826}{unsigned int {\bfseries nb\-Points\-Total}}\label{structSolution_a25fcaf3e9fa7e1a59a149f959f185826}

\end{DoxyCompactItemize}


La documentation de cette structure a été générée à partir du fichier suivant \-:\begin{DoxyCompactItemize}
\item 
solution.\-h\end{DoxyCompactItemize}

\chapter{Documentation des fichiers}
\hypertarget{dictionnaire_8c}{
\section{Référence du fichier dictionnaire.c}
\label{dictionnaire_8c}\index{dictionnaire.c@{dictionnaire.c}}
}


Gestion du dictionnaire Fonctions de gestion du dictionnaire, le dictionnaire doit être dans un fichier texte avec un mot différent par ligne.  


{\ttfamily \#include $<$string.h$>$}\par
{\ttfamily \#include \char`\"{}dictionnaire.h\char`\"{}}\par
{\ttfamily \#include \char`\"{}util.h\char`\"{}}\par
\subsection*{Fonctions}
\begin{DoxyCompactItemize}
\item 
\hyperlink{structDico}{Dico} \hyperlink{dictionnaire_8c_aaff8a3a953fc8b9b360ab39040ff3e5f}{dictionnaire\_\-nouveau} (const char $\ast$pNomFichier)
\begin{DoxyCompactList}\small\item\em Créer un nouveau dictionnaire à partir d'un fichier. \item\end{DoxyCompactList}\item 
int \hyperlink{dictionnaire_8c_a168479ad26583f7fe844a1341a57df1d}{dictonnaire\_\-motDansDico} (\hyperlink{structDico}{Dico} pDictionnaire, char $\ast$pMot)
\begin{DoxyCompactList}\small\item\em Permet de savoir si un mot est présent dans le dictionnaire. \item\end{DoxyCompactList}\item 
void \hyperlink{dictionnaire_8c_a61286b4c5730255db1df056e375d406e}{dictionnaire\_\-rechercheDichotomique} (\hyperlink{structDico}{Dico} pDictionnaire, char $\ast$pMotAChercher, char $\ast$pMotLePlusProche)
\begin{DoxyCompactList}\small\item\em Effectue une recherche dichotomique de pMotAChercher dans le dictionnaire. \item\end{DoxyCompactList}\end{DoxyCompactItemize}


\subsection{Description détaillée}
Gestion du dictionnaire Fonctions de gestion du dictionnaire, le dictionnaire doit être dans un fichier texte avec un mot différent par ligne. 

\subsection{Documentation des fonctions}
\hypertarget{dictionnaire_8c_aaff8a3a953fc8b9b360ab39040ff3e5f}{
\index{dictionnaire.c@{dictionnaire.c}!dictionnaire\_\-nouveau@{dictionnaire\_\-nouveau}}
\index{dictionnaire\_\-nouveau@{dictionnaire\_\-nouveau}!dictionnaire.c@{dictionnaire.c}}
\subsubsection[{dictionnaire\_\-nouveau}]{\setlength{\rightskip}{0pt plus 5cm}{\bf Dico} dictionnaire\_\-nouveau (
\begin{DoxyParamCaption}
\item[{const char $\ast$}]{ pNomFichier}
\end{DoxyParamCaption}
)}}
\label{dictionnaire_8c_aaff8a3a953fc8b9b360ab39040ff3e5f}


Créer un nouveau dictionnaire à partir d'un fichier. 


\begin{DoxyParams}{Paramètres}
\item[{\em pNomFichier}]Le fichier contenant les mots du dictionnaire. \end{DoxyParams}
\begin{DoxyReturn}{Renvoie}
Le nouveau dictionnaire 
\end{DoxyReturn}
\hypertarget{dictionnaire_8c_a61286b4c5730255db1df056e375d406e}{
\index{dictionnaire.c@{dictionnaire.c}!dictionnaire\_\-rechercheDichotomique@{dictionnaire\_\-rechercheDichotomique}}
\index{dictionnaire\_\-rechercheDichotomique@{dictionnaire\_\-rechercheDichotomique}!dictionnaire.c@{dictionnaire.c}}
\subsubsection[{dictionnaire\_\-rechercheDichotomique}]{\setlength{\rightskip}{0pt plus 5cm}void dictionnaire\_\-rechercheDichotomique (
\begin{DoxyParamCaption}
\item[{{\bf Dico}}]{ pDictionnaire, }
\item[{char $\ast$}]{ pMotAChercher, }
\item[{char $\ast$}]{ pMotLePlusProche}
\end{DoxyParamCaption}
)}}
\label{dictionnaire_8c_a61286b4c5730255db1df056e375d406e}


Effectue une recherche dichotomique de pMotAChercher dans le dictionnaire. 


\begin{DoxyParams}{Paramètres}
\item[{\em pDictionnaire}]Le dictionnaire dans lequel chercher \item[{\em pMotAChercher}]Le mot à chercher \item[{\em pMotLePlusProche}]Le mot le plus proche trouvé, si pMotLePlusProche == pMotAChercher, il est présent dans le dictionnaire \end{DoxyParams}
\hypertarget{dictionnaire_8c_a168479ad26583f7fe844a1341a57df1d}{
\index{dictionnaire.c@{dictionnaire.c}!dictonnaire\_\-motDansDico@{dictonnaire\_\-motDansDico}}
\index{dictonnaire\_\-motDansDico@{dictonnaire\_\-motDansDico}!dictionnaire.c@{dictionnaire.c}}
\subsubsection[{dictonnaire\_\-motDansDico}]{\setlength{\rightskip}{0pt plus 5cm}int dictonnaire\_\-motDansDico (
\begin{DoxyParamCaption}
\item[{{\bf Dico}}]{ pDictionnaire, }
\item[{char $\ast$}]{ pMot}
\end{DoxyParamCaption}
)}}
\label{dictionnaire_8c_a168479ad26583f7fe844a1341a57df1d}


Permet de savoir si un mot est présent dans le dictionnaire. 


\begin{DoxyParams}{Paramètres}
\item[{\em pDictionnaire}]Le dictionnaire dans lequel chercher \item[{\em pMot}]Le mot à chercher \end{DoxyParams}
\begin{DoxyReturn}{Renvoie}
0 si le mot est absent, 1 si des mots commencent par pMot et 10 si le mot exact est trouvé. 
\end{DoxyReturn}

\hypertarget{interfaceNcurses_8c}{\section{Référence du fichier interface\-Ncurses.\-c}
\label{interfaceNcurses_8c}\index{interface\-Ncurses.\-c@{interface\-Ncurses.\-c}}
}


Affichage avec la bibliothèque Ncurses Affiche et permet de jouer au jeu de Boggle avec un affichage utilisant la bibliothèque Ncurses.  


{\ttfamily \#include $<$ncurses.\-h$>$}\\*
{\ttfamily \#include $<$panel.\-h$>$}\\*
{\ttfamily \#include $<$menu.\-h$>$}\\*
{\ttfamily \#include $<$stdlib.\-h$>$}\\*
{\ttfamily \#include $<$string.\-h$>$}\\*
{\ttfamily \#include \char`\"{}interface\-Ncurses.\-h\char`\"{}}\\*
{\ttfamily \#include \char`\"{}plateau.\-h\char`\"{}}\\*
{\ttfamily \#include \char`\"{}jeu.\-h\char`\"{}}\\*
{\ttfamily \#include \char`\"{}util.\-h\char`\"{}}\\*
\subsection*{Fonctions}
\begin{DoxyCompactItemize}
\item 
int \hyperlink{interfaceNcurses_8c_a59c689b075d61c9dbb8c752a1c85a56e}{fct\-Menu} (char $\ast$name)
\begin{DoxyCompactList}\small\item\em Retourne le numéro de l'item appelé \end{DoxyCompactList}\item 
void \hyperlink{interfaceNcurses_8c_a8e2f090f4d8640151c9a0957911d99e5}{interface\-Ncurses\-\_\-afficher\-Titre} (const char $\ast$p\-Title)
\begin{DoxyCompactList}\small\item\em Afficher un titre à la fenêtre. \end{DoxyCompactList}\item 
void \hyperlink{interfaceNcurses_8c_a27c4bdc3571ec76bc5f31b11847c0680}{interface\-Ncurses\-\_\-afficher\-Solution} (const \hyperlink{structSolution}{Solution} p\-Solution)
\begin{DoxyCompactList}\small\item\em Affiche la solution passée en paramètre. \end{DoxyCompactList}\item 
W\-I\-N\-D\-O\-W $\ast$ \hyperlink{interfaceNcurses_8c_a11139ea3bdd922798149d2fa93e93457}{interface\-Ncurses\-\_\-initialiser} (void)
\begin{DoxyCompactList}\small\item\em Initialise l'interface Ncurses. \end{DoxyCompactList}\item 
void \hyperlink{interfaceNcurses_8c_a7f08cba800239f2e0bdfa3ed50303595}{interface\-Ncurses\-\_\-afficher\-Grille} (const \hyperlink{structPlateau}{Plateau} p\-Plateau, const \hyperlink{structCase}{Case} p\-Selected\-Case, const \hyperlink{structCase}{Case} $\ast$p\-Used\-Case, const int p\-Lg\-Used\-Case)
\begin{DoxyCompactList}\small\item\em Affiche la grille, composée de couleurs en fonctions des différentes cases. \end{DoxyCompactList}\item 
void \hyperlink{interfaceNcurses_8c_a8107452f526ae2b4bb7db9daadd39fa9}{interface\-Ncurses\-\_\-terminer} (W\-I\-N\-D\-O\-W $\ast$fenetre, \hyperlink{structJeu}{Jeu} p\-Jeu)
\begin{DoxyCompactList}\small\item\em Termine le jeu. \end{DoxyCompactList}\item 
void \hyperlink{interfaceNcurses_8c_ad8d628ccec45bfb57ace978b88a6dd0e}{jeu\-\_\-lancer\-Mode\-Ncurses} (\hyperlink{structJeu}{Jeu} p\-Jeu)
\begin{DoxyCompactList}\small\item\em Lance le mode Ncurses. \end{DoxyCompactList}\item 
void \hyperlink{interfaceNcurses_8c_ad154999bd280bf4911c014bcaf00670f}{interface\-Ncurses\-\_\-afficher\-Boite\-Dialogue} (const \hyperlink{structJeu}{Jeu} p\-Jeu)
\begin{DoxyCompactList}\small\item\em Affiche une boite de dialogue de fin de \hyperlink{structJeu}{Jeu}. \end{DoxyCompactList}\item 
void \hyperlink{interfaceNcurses_8c_aff38ebc6fd2d7cf8263278889833b289}{interface\-Ncurses\-\_\-afficher\-Fenetre\-Jeu} (const \hyperlink{structJeu}{Jeu} p\-Jeu, char $\ast$p\-Mot, const \hyperlink{structCase}{Case} p\-Selected\-Case, \hyperlink{structCase}{Case} $\ast$p\-Used\-Case)
\begin{DoxyCompactList}\small\item\em Affiche la fenêtre permettant de jouer. \end{DoxyCompactList}\item 
\hypertarget{interfaceNcurses_8c_ab009b3ac48e7a5753bb4473b2c2c7283}{void {\bfseries win\-\_\-show} (W\-I\-N\-D\-O\-W $\ast$win, char $\ast$label, \hyperlink{structJeu}{Jeu} p\-Jeu)}\label{interfaceNcurses_8c_ab009b3ac48e7a5753bb4473b2c2c7283}

\item 
void \hyperlink{interfaceNcurses_8c_a013c16f098ebd2e99f3ef5604235f78c}{interface\-Ncurses\-\_\-menu} (W\-I\-N\-D\-O\-W $\ast$p\-Dialog\-Box\-Win, const \hyperlink{structJeu}{Jeu} p\-Jeu)
\begin{DoxyCompactList}\small\item\em Affiche un menu permettant de choisir entre quitter et afficher la solution. \end{DoxyCompactList}\item 
void \hyperlink{interfaceNcurses_8c_aafd8deb32c222d5c6fa5f976118332bc}{interface\-Ncurses\-\_\-afficher\-Fenetre\-Solution} (const \hyperlink{structJeu}{Jeu} p\-Jeu)
\begin{DoxyCompactList}\small\item\em Affiche la fenêtre avec la solution complète du jeu. \end{DoxyCompactList}\item 
\hypertarget{interfaceNcurses_8c_aa956594ead8007cab451cc96c8813205}{void {\bfseries print\-\_\-in\-\_\-middle} (W\-I\-N\-D\-O\-W $\ast$win, int starty, int startx, int width, char $\ast$string, \hyperlink{structJeu}{Jeu} p\-Jeu)}\label{interfaceNcurses_8c_aa956594ead8007cab451cc96c8813205}

\end{DoxyCompactItemize}


\subsection{Description détaillée}
Affichage avec la bibliothèque Ncurses Affiche et permet de jouer au jeu de Boggle avec un affichage utilisant la bibliothèque Ncurses. 

\subsection{Documentation des fonctions}
\hypertarget{interfaceNcurses_8c_a59c689b075d61c9dbb8c752a1c85a56e}{\index{interface\-Ncurses.\-c@{interface\-Ncurses.\-c}!fct\-Menu@{fct\-Menu}}
\index{fct\-Menu@{fct\-Menu}!interfaceNcurses.c@{interface\-Ncurses.\-c}}
\subsubsection[{fct\-Menu}]{\setlength{\rightskip}{0pt plus 5cm}int fct\-Menu (
\begin{DoxyParamCaption}
\item[{char $\ast$}]{name}
\end{DoxyParamCaption}
)}}\label{interfaceNcurses_8c_a59c689b075d61c9dbb8c752a1c85a56e}


Retourne le numéro de l'item appelé 


\begin{DoxyParams}{Paramètres}
{\em Le} & nom de l'item \\
\hline
\end{DoxyParams}
\begin{DoxyReturn}{Renvoie}
Le numéro de l'item appelé 
\end{DoxyReturn}
\hypertarget{interfaceNcurses_8c_ad154999bd280bf4911c014bcaf00670f}{\index{interface\-Ncurses.\-c@{interface\-Ncurses.\-c}!interface\-Ncurses\-\_\-afficher\-Boite\-Dialogue@{interface\-Ncurses\-\_\-afficher\-Boite\-Dialogue}}
\index{interface\-Ncurses\-\_\-afficher\-Boite\-Dialogue@{interface\-Ncurses\-\_\-afficher\-Boite\-Dialogue}!interfaceNcurses.c@{interface\-Ncurses.\-c}}
\subsubsection[{interface\-Ncurses\-\_\-afficher\-Boite\-Dialogue}]{\setlength{\rightskip}{0pt plus 5cm}void interface\-Ncurses\-\_\-afficher\-Boite\-Dialogue (
\begin{DoxyParamCaption}
\item[{const {\bf Jeu}}]{p\-Jeu}
\end{DoxyParamCaption}
)}}\label{interfaceNcurses_8c_ad154999bd280bf4911c014bcaf00670f}


Affiche une boite de dialogue de fin de \hyperlink{structJeu}{Jeu}. 


\begin{DoxyParams}{Paramètres}
{\em p\-Jeu} & Le jeu \\
\hline
\end{DoxyParams}
\hypertarget{interfaceNcurses_8c_aff38ebc6fd2d7cf8263278889833b289}{\index{interface\-Ncurses.\-c@{interface\-Ncurses.\-c}!interface\-Ncurses\-\_\-afficher\-Fenetre\-Jeu@{interface\-Ncurses\-\_\-afficher\-Fenetre\-Jeu}}
\index{interface\-Ncurses\-\_\-afficher\-Fenetre\-Jeu@{interface\-Ncurses\-\_\-afficher\-Fenetre\-Jeu}!interfaceNcurses.c@{interface\-Ncurses.\-c}}
\subsubsection[{interface\-Ncurses\-\_\-afficher\-Fenetre\-Jeu}]{\setlength{\rightskip}{0pt plus 5cm}void interface\-Ncurses\-\_\-afficher\-Fenetre\-Jeu (
\begin{DoxyParamCaption}
\item[{const {\bf Jeu}}]{p\-Jeu, }
\item[{char $\ast$}]{p\-Mot, }
\item[{const {\bf Case}}]{p\-Selected\-Case, }
\item[{{\bf Case} $\ast$}]{p\-Used\-Case}
\end{DoxyParamCaption}
)}}\label{interfaceNcurses_8c_aff38ebc6fd2d7cf8263278889833b289}


Affiche la fenêtre permettant de jouer. 


\begin{DoxyParams}{Paramètres}
{\em p\-Jeu} & Le jeu \\
\hline
{\em p\-Mot} & Le mot courant \\
\hline
{\em p\-Selected\-Case} & La case séléctionnée \\
\hline
{\em p\-Used\-Case} & Les cases utilisés pour créer le mot p\-Mot \\
\hline
\end{DoxyParams}
\hypertarget{interfaceNcurses_8c_aafd8deb32c222d5c6fa5f976118332bc}{\index{interface\-Ncurses.\-c@{interface\-Ncurses.\-c}!interface\-Ncurses\-\_\-afficher\-Fenetre\-Solution@{interface\-Ncurses\-\_\-afficher\-Fenetre\-Solution}}
\index{interface\-Ncurses\-\_\-afficher\-Fenetre\-Solution@{interface\-Ncurses\-\_\-afficher\-Fenetre\-Solution}!interfaceNcurses.c@{interface\-Ncurses.\-c}}
\subsubsection[{interface\-Ncurses\-\_\-afficher\-Fenetre\-Solution}]{\setlength{\rightskip}{0pt plus 5cm}void interface\-Ncurses\-\_\-afficher\-Fenetre\-Solution (
\begin{DoxyParamCaption}
\item[{const {\bf Jeu}}]{p\-Jeu}
\end{DoxyParamCaption}
)}}\label{interfaceNcurses_8c_aafd8deb32c222d5c6fa5f976118332bc}


Affiche la fenêtre avec la solution complète du jeu. 


\begin{DoxyParams}{Paramètres}
{\em p\-Jeu} & Le jeu \\
\hline
\end{DoxyParams}
\hypertarget{interfaceNcurses_8c_a7f08cba800239f2e0bdfa3ed50303595}{\index{interface\-Ncurses.\-c@{interface\-Ncurses.\-c}!interface\-Ncurses\-\_\-afficher\-Grille@{interface\-Ncurses\-\_\-afficher\-Grille}}
\index{interface\-Ncurses\-\_\-afficher\-Grille@{interface\-Ncurses\-\_\-afficher\-Grille}!interfaceNcurses.c@{interface\-Ncurses.\-c}}
\subsubsection[{interface\-Ncurses\-\_\-afficher\-Grille}]{\setlength{\rightskip}{0pt plus 5cm}void interface\-Ncurses\-\_\-afficher\-Grille (
\begin{DoxyParamCaption}
\item[{const {\bf Plateau}}]{p\-Plateau, }
\item[{const {\bf Case}}]{p\-Selected\-Case, }
\item[{const {\bf Case} $\ast$}]{p\-Used\-Case, }
\item[{const int}]{p\-Lg\-Used\-Case}
\end{DoxyParamCaption}
)}}\label{interfaceNcurses_8c_a7f08cba800239f2e0bdfa3ed50303595}


Affiche la grille, composée de couleurs en fonctions des différentes cases. 


\begin{DoxyParams}{Paramètres}
{\em p\-Plateau} & La grille à afficher \\
\hline
{\em p\-Selected\-Case} & La case séléctionnée \\
\hline
{\em p\-Used\-Case} & Les cases utilisés pour le mot courant \\
\hline
{\em p\-Lg\-Used\-Case} & La longueur du mot courant \\
\hline
\end{DoxyParams}
\hypertarget{interfaceNcurses_8c_a27c4bdc3571ec76bc5f31b11847c0680}{\index{interface\-Ncurses.\-c@{interface\-Ncurses.\-c}!interface\-Ncurses\-\_\-afficher\-Solution@{interface\-Ncurses\-\_\-afficher\-Solution}}
\index{interface\-Ncurses\-\_\-afficher\-Solution@{interface\-Ncurses\-\_\-afficher\-Solution}!interfaceNcurses.c@{interface\-Ncurses.\-c}}
\subsubsection[{interface\-Ncurses\-\_\-afficher\-Solution}]{\setlength{\rightskip}{0pt plus 5cm}void interface\-Ncurses\-\_\-afficher\-Solution (
\begin{DoxyParamCaption}
\item[{const {\bf Solution}}]{p\-Solution}
\end{DoxyParamCaption}
)}}\label{interfaceNcurses_8c_a27c4bdc3571ec76bc5f31b11847c0680}


Affiche la solution passée en paramètre. 


\begin{DoxyParams}{Paramètres}
{\em p\-Solution} & La solution à afficher \\
\hline
\end{DoxyParams}
\hypertarget{interfaceNcurses_8c_a8e2f090f4d8640151c9a0957911d99e5}{\index{interface\-Ncurses.\-c@{interface\-Ncurses.\-c}!interface\-Ncurses\-\_\-afficher\-Titre@{interface\-Ncurses\-\_\-afficher\-Titre}}
\index{interface\-Ncurses\-\_\-afficher\-Titre@{interface\-Ncurses\-\_\-afficher\-Titre}!interfaceNcurses.c@{interface\-Ncurses.\-c}}
\subsubsection[{interface\-Ncurses\-\_\-afficher\-Titre}]{\setlength{\rightskip}{0pt plus 5cm}void interface\-Ncurses\-\_\-afficher\-Titre (
\begin{DoxyParamCaption}
\item[{const char $\ast$}]{p\-Title}
\end{DoxyParamCaption}
)}}\label{interfaceNcurses_8c_a8e2f090f4d8640151c9a0957911d99e5}


Afficher un titre à la fenêtre. 


\begin{DoxyParams}{Paramètres}
{\em p\-Title} & Le titre à afficher \\
\hline
\end{DoxyParams}
\hypertarget{interfaceNcurses_8c_a11139ea3bdd922798149d2fa93e93457}{\index{interface\-Ncurses.\-c@{interface\-Ncurses.\-c}!interface\-Ncurses\-\_\-initialiser@{interface\-Ncurses\-\_\-initialiser}}
\index{interface\-Ncurses\-\_\-initialiser@{interface\-Ncurses\-\_\-initialiser}!interfaceNcurses.c@{interface\-Ncurses.\-c}}
\subsubsection[{interface\-Ncurses\-\_\-initialiser}]{\setlength{\rightskip}{0pt plus 5cm}W\-I\-N\-D\-O\-W$\ast$ interface\-Ncurses\-\_\-initialiser (
\begin{DoxyParamCaption}
\item[{void}]{}
\end{DoxyParamCaption}
)}}\label{interfaceNcurses_8c_a11139ea3bdd922798149d2fa93e93457}


Initialise l'interface Ncurses. 

\begin{DoxyReturn}{Renvoie}
La fenêtre créée 
\end{DoxyReturn}
\hypertarget{interfaceNcurses_8c_a013c16f098ebd2e99f3ef5604235f78c}{\index{interface\-Ncurses.\-c@{interface\-Ncurses.\-c}!interface\-Ncurses\-\_\-menu@{interface\-Ncurses\-\_\-menu}}
\index{interface\-Ncurses\-\_\-menu@{interface\-Ncurses\-\_\-menu}!interfaceNcurses.c@{interface\-Ncurses.\-c}}
\subsubsection[{interface\-Ncurses\-\_\-menu}]{\setlength{\rightskip}{0pt plus 5cm}void interface\-Ncurses\-\_\-menu (
\begin{DoxyParamCaption}
\item[{W\-I\-N\-D\-O\-W $\ast$}]{p\-Dialog\-Box\-Win, }
\item[{const {\bf Jeu}}]{p\-Jeu}
\end{DoxyParamCaption}
)}}\label{interfaceNcurses_8c_a013c16f098ebd2e99f3ef5604235f78c}


Affiche un menu permettant de choisir entre quitter et afficher la solution. 


\begin{DoxyParams}{Paramètres}
{\em p\-Dialog\-Box\-Win} & La fenêtre danslequel s'affiche le menu \\
\hline
{\em p\-Jeu} & Le jeu \\
\hline
\end{DoxyParams}
\begin{DoxyReturn}{Renvoie}

\end{DoxyReturn}
\hypertarget{interfaceNcurses_8c_a8107452f526ae2b4bb7db9daadd39fa9}{\index{interface\-Ncurses.\-c@{interface\-Ncurses.\-c}!interface\-Ncurses\-\_\-terminer@{interface\-Ncurses\-\_\-terminer}}
\index{interface\-Ncurses\-\_\-terminer@{interface\-Ncurses\-\_\-terminer}!interfaceNcurses.c@{interface\-Ncurses.\-c}}
\subsubsection[{interface\-Ncurses\-\_\-terminer}]{\setlength{\rightskip}{0pt plus 5cm}void interface\-Ncurses\-\_\-terminer (
\begin{DoxyParamCaption}
\item[{W\-I\-N\-D\-O\-W $\ast$}]{fenetre, }
\item[{{\bf Jeu}}]{p\-Jeu}
\end{DoxyParamCaption}
)}}\label{interfaceNcurses_8c_a8107452f526ae2b4bb7db9daadd39fa9}


Termine le jeu. 


\begin{DoxyParams}{Paramètres}
{\em fenetre} & La fenêtre à détruire \\
\hline
{\em p\-Jeu} & le jeu à terminer \\
\hline
\end{DoxyParams}
\hypertarget{interfaceNcurses_8c_ad8d628ccec45bfb57ace978b88a6dd0e}{\index{interface\-Ncurses.\-c@{interface\-Ncurses.\-c}!jeu\-\_\-lancer\-Mode\-Ncurses@{jeu\-\_\-lancer\-Mode\-Ncurses}}
\index{jeu\-\_\-lancer\-Mode\-Ncurses@{jeu\-\_\-lancer\-Mode\-Ncurses}!interfaceNcurses.c@{interface\-Ncurses.\-c}}
\subsubsection[{jeu\-\_\-lancer\-Mode\-Ncurses}]{\setlength{\rightskip}{0pt plus 5cm}void jeu\-\_\-lancer\-Mode\-Ncurses (
\begin{DoxyParamCaption}
\item[{{\bf Jeu}}]{p\-Jeu}
\end{DoxyParamCaption}
)}}\label{interfaceNcurses_8c_ad8d628ccec45bfb57ace978b88a6dd0e}


Lance le mode Ncurses. 


\begin{DoxyParams}{Paramètres}
{\em p\-Jeu} & Le jeu à lancer \\
\hline
\end{DoxyParams}

\hypertarget{jeu_8c}{\section{Référence du fichier jeu.\-c}
\label{jeu_8c}\index{jeu.\-c@{jeu.\-c}}
}


Gestion du jeu de Boggle Les fonctions permettant de gérer le \hyperlink{structJeu}{Jeu} de Boggle.  


{\ttfamily \#include $<$stdlib.\-h$>$}\\*
{\ttfamily \#include $<$string.\-h$>$}\\*
{\ttfamily \#include \char`\"{}util.\-h\char`\"{}}\\*
{\ttfamily \#include \char`\"{}resolveur.\-h\char`\"{}}\\*
{\ttfamily \#include \char`\"{}jeu.\-h\char`\"{}}\\*
{\ttfamily \#include \char`\"{}plateau.\-h\char`\"{}}\\*
{\ttfamily \#include \char`\"{}dictionnaire.\-h\char`\"{}}\\*
{\ttfamily \#include \char`\"{}interface\-Texte.\-h\char`\"{}}\\*
\subsection*{Fonctions}
\begin{DoxyCompactItemize}
\item 
time\-\_\-t \hyperlink{jeu_8c_a59f32d1375181989b1aa94b838631929}{jeu\-\_\-temps\-Restant} (const \hyperlink{structJeu}{Jeu} p\-Jeu)
\begin{DoxyCompactList}\small\item\em Retourne le nombre de secondes restantes avant la fin du jeu. \end{DoxyCompactList}\item 
bool \hyperlink{jeu_8c_a856c5fb8ebf30b42d3b0dccc0b3ad2e4}{jeu\-\_\-compteur\-Claque} (const \hyperlink{structJeu}{Jeu} p\-Jeu)
\begin{DoxyCompactList}\small\item\em Retourne vrai si le temps est écoulé faux sinon. \end{DoxyCompactList}\item 
\hyperlink{structJeu}{Jeu} \hyperlink{jeu_8c_a5fcc7a42a3e56660da1479ac8c859d16}{jeu\-\_\-nouveau} (const char $\ast$p\-Nom\-Dico, const \hyperlink{structCouple}{Couple} p\-Taille\-Plateau, const unsigned int p\-Lg\-Timer)
\begin{DoxyCompactList}\small\item\em Créer un nouveau jeu. \end{DoxyCompactList}\item 
void \hyperlink{jeu_8c_ac9f225a777f911ea04166e4d8b8323b2}{jeu\-\_\-lancer} (\hyperlink{structJeu}{Jeu} $\ast$p\-Jeu)
\begin{DoxyCompactList}\small\item\em Lance le jeu. \end{DoxyCompactList}\item 
\-\_\-\-Bool \hyperlink{jeu_8c_aee397c079a99210e5594e36b08db11c4}{jeu\-\_\-proposer\-Mot} (\hyperlink{structJeu}{Jeu} $\ast$p\-Jeu, const char $\ast$p\-Mot)
\begin{DoxyCompactList}\small\item\em Proposer un mot. \end{DoxyCompactList}\item 
void \hyperlink{jeu_8c_a112bcfc7f5159ea64a21cacb92c944ea}{jeu\-\_\-stopper} (\hyperlink{structJeu}{Jeu} p\-Jeu)
\begin{DoxyCompactList}\small\item\em Fin du jeu. \end{DoxyCompactList}\end{DoxyCompactItemize}


\subsection{Description détaillée}
Gestion du jeu de Boggle Les fonctions permettant de gérer le \hyperlink{structJeu}{Jeu} de Boggle. \begin{DoxySeeAlso}{Voir également}
\hyperlink{structPlateau}{Plateau}, Resolveur, \hyperlink{structDico}{Dico}, \hyperlink{structSolution}{Solution} 
\end{DoxySeeAlso}


\subsection{Documentation des fonctions}
\hypertarget{jeu_8c_a856c5fb8ebf30b42d3b0dccc0b3ad2e4}{\index{jeu.\-c@{jeu.\-c}!jeu\-\_\-compteur\-Claque@{jeu\-\_\-compteur\-Claque}}
\index{jeu\-\_\-compteur\-Claque@{jeu\-\_\-compteur\-Claque}!jeu.c@{jeu.\-c}}
\subsubsection[{jeu\-\_\-compteur\-Claque}]{\setlength{\rightskip}{0pt plus 5cm}bool jeu\-\_\-compteur\-Claque (
\begin{DoxyParamCaption}
\item[{const {\bf Jeu}}]{p\-Jeu}
\end{DoxyParamCaption}
)}}\label{jeu_8c_a856c5fb8ebf30b42d3b0dccc0b3ad2e4}


Retourne vrai si le temps est écoulé faux sinon. 


\begin{DoxyParams}{Paramètres}
{\em p\-Jeu} & Le jeu pourlequel on veut saoir si le temps est écoulé \\
\hline
\end{DoxyParams}
\begin{DoxyReturn}{Renvoie}
Vrai si le temps est écoulé faux sinon 
\end{DoxyReturn}
\hypertarget{jeu_8c_ac9f225a777f911ea04166e4d8b8323b2}{\index{jeu.\-c@{jeu.\-c}!jeu\-\_\-lancer@{jeu\-\_\-lancer}}
\index{jeu\-\_\-lancer@{jeu\-\_\-lancer}!jeu.c@{jeu.\-c}}
\subsubsection[{jeu\-\_\-lancer}]{\setlength{\rightskip}{0pt plus 5cm}void jeu\-\_\-lancer (
\begin{DoxyParamCaption}
\item[{{\bf Jeu} $\ast$}]{p\-Jeu}
\end{DoxyParamCaption}
)}}\label{jeu_8c_ac9f225a777f911ea04166e4d8b8323b2}


Lance le jeu. 


\begin{DoxyParams}{Paramètres}
{\em p\-Jeu} & Le jeu à lancer \\
\hline
\end{DoxyParams}
\hypertarget{jeu_8c_a5fcc7a42a3e56660da1479ac8c859d16}{\index{jeu.\-c@{jeu.\-c}!jeu\-\_\-nouveau@{jeu\-\_\-nouveau}}
\index{jeu\-\_\-nouveau@{jeu\-\_\-nouveau}!jeu.c@{jeu.\-c}}
\subsubsection[{jeu\-\_\-nouveau}]{\setlength{\rightskip}{0pt plus 5cm}{\bf Jeu} jeu\-\_\-nouveau (
\begin{DoxyParamCaption}
\item[{const char $\ast$}]{p\-Nom\-Dico, }
\item[{const {\bf Couple}}]{p\-Taille\-Plateau, }
\item[{const unsigned int}]{p\-Lg\-Timer}
\end{DoxyParamCaption}
)}}\label{jeu_8c_a5fcc7a42a3e56660da1479ac8c859d16}


Créer un nouveau jeu. 


\begin{DoxyParams}{Paramètres}
{\em p\-Nom\-Dico} & Le dictionnaire utilisé dans le \hyperlink{structJeu}{Jeu} \\
\hline
{\em p\-Taille\-Plateau} & La taille du plateau à créer \\
\hline
{\em p\-Lg\-Timer} & La longeur du timer en secondes \\
\hline
\end{DoxyParams}
\begin{DoxyReturn}{Renvoie}
Le nouveau \hyperlink{structJeu}{Jeu} 
\end{DoxyReturn}
\hypertarget{jeu_8c_aee397c079a99210e5594e36b08db11c4}{\index{jeu.\-c@{jeu.\-c}!jeu\-\_\-proposer\-Mot@{jeu\-\_\-proposer\-Mot}}
\index{jeu\-\_\-proposer\-Mot@{jeu\-\_\-proposer\-Mot}!jeu.c@{jeu.\-c}}
\subsubsection[{jeu\-\_\-proposer\-Mot}]{\setlength{\rightskip}{0pt plus 5cm}\-\_\-\-Bool jeu\-\_\-proposer\-Mot (
\begin{DoxyParamCaption}
\item[{{\bf Jeu} $\ast$}]{p\-Jeu, }
\item[{const char $\ast$}]{p\-Mot}
\end{DoxyParamCaption}
)}}\label{jeu_8c_aee397c079a99210e5594e36b08db11c4}


Proposer un mot. 


\begin{DoxyParams}{Paramètres}
{\em p\-Jeu} & Le jeu pour lequel on veut proposer un mot \\
\hline
{\em p\-Mot} & Le mot proposé \\
\hline
\end{DoxyParams}
\begin{DoxyReturn}{Renvoie}
Vrai si le mot est validé 
\end{DoxyReturn}
\hypertarget{jeu_8c_a112bcfc7f5159ea64a21cacb92c944ea}{\index{jeu.\-c@{jeu.\-c}!jeu\-\_\-stopper@{jeu\-\_\-stopper}}
\index{jeu\-\_\-stopper@{jeu\-\_\-stopper}!jeu.c@{jeu.\-c}}
\subsubsection[{jeu\-\_\-stopper}]{\setlength{\rightskip}{0pt plus 5cm}void jeu\-\_\-stopper (
\begin{DoxyParamCaption}
\item[{{\bf Jeu}}]{p\-Jeu}
\end{DoxyParamCaption}
)}}\label{jeu_8c_a112bcfc7f5159ea64a21cacb92c944ea}


Fin du jeu. 


\begin{DoxyParams}{Paramètres}
{\em p\-Jeu} & Le jeu à terminer \\
\hline
\end{DoxyParams}
\hypertarget{jeu_8c_a59f32d1375181989b1aa94b838631929}{\index{jeu.\-c@{jeu.\-c}!jeu\-\_\-temps\-Restant@{jeu\-\_\-temps\-Restant}}
\index{jeu\-\_\-temps\-Restant@{jeu\-\_\-temps\-Restant}!jeu.c@{jeu.\-c}}
\subsubsection[{jeu\-\_\-temps\-Restant}]{\setlength{\rightskip}{0pt plus 5cm}time\-\_\-t jeu\-\_\-temps\-Restant (
\begin{DoxyParamCaption}
\item[{const {\bf Jeu}}]{p\-Jeu}
\end{DoxyParamCaption}
)}}\label{jeu_8c_a59f32d1375181989b1aa94b838631929}


Retourne le nombre de secondes restantes avant la fin du jeu. 


\begin{DoxyParams}{Paramètres}
{\em p\-Jeu} & Le jeu pour lequel on veut savoir le temps restant \\
\hline
\end{DoxyParams}
\begin{DoxyReturn}{Renvoie}
Le nombre de secondes restantes 
\end{DoxyReturn}

\hypertarget{plateau_8c}{
\section{Référence du fichier plateau.c}
\label{plateau_8c}\index{plateau.c@{plateau.c}}
}


Gestion du plateau de \hyperlink{structJeu}{Jeu} Fonctions se rapportant à un plateau de jeu, sera sert à gérer la grille de Boggle.  


{\ttfamily \#include $<$stdlib.h$>$}\par
{\ttfamily \#include \char`\"{}solution.h\char`\"{}}\par
{\ttfamily \#include \char`\"{}plateau.h\char`\"{}}\par
{\ttfamily \#include \char`\"{}util.h\char`\"{}}\par
\subsection*{Fonctions}
\begin{DoxyCompactItemize}
\item 
\hyperlink{structPlateau}{Plateau} \hyperlink{plateau_8c_a227b9ee882d6ee30797b4dd978a26633}{plateau\_\-nouveau} (const unsigned char pTailleGrille)
\begin{DoxyCompactList}\small\item\em Créer un nouveau plateau. \item\end{DoxyCompactList}\item 
void \hyperlink{plateau_8c_afeb9fd22a43c9a29c9815be5ffbc254d}{plateau\_\-detruire} (\hyperlink{structPlateau}{Plateau} $\ast$pPlateau)
\begin{DoxyCompactList}\small\item\em Détruit le plateau pPlateau. \item\end{DoxyCompactList}\item 
void \hyperlink{plateau_8c_a23a1db5b648d7483da8619e880506725}{plateau\_\-remplirGrilleAleatoire} (\hyperlink{structPlateau}{Plateau} $\ast$pPlateau)
\begin{DoxyCompactList}\small\item\em Rempli une grille aléatoirement dans le plateau. \item\end{DoxyCompactList}\item 
void \hyperlink{plateau_8c_aa2bb750ea8570ed32eb894c7249d2e35}{plateau\_\-remplirGrillePredefinie} (\hyperlink{structPlateau}{Plateau} $\ast$pPlateau)
\begin{DoxyCompactList}\small\item\em Rempli le plateau avec une grille prédéfinie de taille 4$\ast$4 E D R C A N V C I R Q A E B R U. \item\end{DoxyCompactList}\item 
char \hyperlink{plateau_8c_a859a80b6e007a4f7e9fff96414f24384}{plateau\_\-choisirLettre} (void)
\begin{DoxyCompactList}\small\item\em Choisi une letre avec la probabilité des lettres en fonctions de leurs importance d'apparition dans la langue française. \item\end{DoxyCompactList}\item 
double \hyperlink{plateau_8c_ac564837afaa7e07ed257caf8639a5879}{plateau\_\-probaLettre} (const char pLettre)
\begin{DoxyCompactList}\small\item\em Retourne la probabilité d'apparition de la lettre passée en paramètre. \item\end{DoxyCompactList}\end{DoxyCompactItemize}


\subsection{Description détaillée}
Gestion du plateau de \hyperlink{structJeu}{Jeu} Fonctions se rapportant à un plateau de jeu, sera sert à gérer la grille de Boggle. 

\subsection{Documentation des fonctions}
\hypertarget{plateau_8c_a859a80b6e007a4f7e9fff96414f24384}{
\index{plateau.c@{plateau.c}!plateau\_\-choisirLettre@{plateau\_\-choisirLettre}}
\index{plateau\_\-choisirLettre@{plateau\_\-choisirLettre}!plateau.c@{plateau.c}}
\subsubsection[{plateau\_\-choisirLettre}]{\setlength{\rightskip}{0pt plus 5cm}char plateau\_\-choisirLettre (
\begin{DoxyParamCaption}
\item[{void}]{}
\end{DoxyParamCaption}
)}}
\label{plateau_8c_a859a80b6e007a4f7e9fff96414f24384}


Choisi une letre avec la probabilité des lettres en fonctions de leurs importance d'apparition dans la langue française. 

\begin{DoxyReturn}{Renvoie}
La lettre choisie 
\end{DoxyReturn}
\hypertarget{plateau_8c_afeb9fd22a43c9a29c9815be5ffbc254d}{
\index{plateau.c@{plateau.c}!plateau\_\-detruire@{plateau\_\-detruire}}
\index{plateau\_\-detruire@{plateau\_\-detruire}!plateau.c@{plateau.c}}
\subsubsection[{plateau\_\-detruire}]{\setlength{\rightskip}{0pt plus 5cm}void plateau\_\-detruire (
\begin{DoxyParamCaption}
\item[{{\bf Plateau} $\ast$}]{ pPlateau}
\end{DoxyParamCaption}
)}}
\label{plateau_8c_afeb9fd22a43c9a29c9815be5ffbc254d}


Détruit le plateau pPlateau. 


\begin{DoxyParams}{Paramètres}
\item[{\em pPlateau}]Le plateau à détruire \end{DoxyParams}
\hypertarget{plateau_8c_a227b9ee882d6ee30797b4dd978a26633}{
\index{plateau.c@{plateau.c}!plateau\_\-nouveau@{plateau\_\-nouveau}}
\index{plateau\_\-nouveau@{plateau\_\-nouveau}!plateau.c@{plateau.c}}
\subsubsection[{plateau\_\-nouveau}]{\setlength{\rightskip}{0pt plus 5cm}{\bf Plateau} plateau\_\-nouveau (
\begin{DoxyParamCaption}
\item[{const unsigned char}]{ pTailleGrille}
\end{DoxyParamCaption}
)}}
\label{plateau_8c_a227b9ee882d6ee30797b4dd978a26633}


Créer un nouveau plateau. 


\begin{DoxyParams}{Paramètres}
\item[{\em pTailleGrille}]La taille de la grille à créer \end{DoxyParams}
\begin{DoxyReturn}{Renvoie}
Le nouveau plateau 
\end{DoxyReturn}
\hypertarget{plateau_8c_ac564837afaa7e07ed257caf8639a5879}{
\index{plateau.c@{plateau.c}!plateau\_\-probaLettre@{plateau\_\-probaLettre}}
\index{plateau\_\-probaLettre@{plateau\_\-probaLettre}!plateau.c@{plateau.c}}
\subsubsection[{plateau\_\-probaLettre}]{\setlength{\rightskip}{0pt plus 5cm}double plateau\_\-probaLettre (
\begin{DoxyParamCaption}
\item[{const char}]{ pLettre}
\end{DoxyParamCaption}
)}}
\label{plateau_8c_ac564837afaa7e07ed257caf8639a5879}


Retourne la probabilité d'apparition de la lettre passée en paramètre. 


\begin{DoxyParams}{Paramètres}
\item[{\em pLettre}]La lettre pour laquelle retourner la probabilité \end{DoxyParams}
\begin{DoxyReturn}{Renvoie}
La probabilité 
\end{DoxyReturn}
\hypertarget{plateau_8c_a23a1db5b648d7483da8619e880506725}{
\index{plateau.c@{plateau.c}!plateau\_\-remplirGrilleAleatoire@{plateau\_\-remplirGrilleAleatoire}}
\index{plateau\_\-remplirGrilleAleatoire@{plateau\_\-remplirGrilleAleatoire}!plateau.c@{plateau.c}}
\subsubsection[{plateau\_\-remplirGrilleAleatoire}]{\setlength{\rightskip}{0pt plus 5cm}void plateau\_\-remplirGrilleAleatoire (
\begin{DoxyParamCaption}
\item[{{\bf Plateau} $\ast$}]{ pPlateau}
\end{DoxyParamCaption}
)}}
\label{plateau_8c_a23a1db5b648d7483da8619e880506725}


Rempli une grille aléatoirement dans le plateau. 


\begin{DoxyParams}{Paramètres}
\item[{\em pPlateau}]Le plateau à remplire \end{DoxyParams}
\hypertarget{plateau_8c_aa2bb750ea8570ed32eb894c7249d2e35}{
\index{plateau.c@{plateau.c}!plateau\_\-remplirGrillePredefinie@{plateau\_\-remplirGrillePredefinie}}
\index{plateau\_\-remplirGrillePredefinie@{plateau\_\-remplirGrillePredefinie}!plateau.c@{plateau.c}}
\subsubsection[{plateau\_\-remplirGrillePredefinie}]{\setlength{\rightskip}{0pt plus 5cm}void plateau\_\-remplirGrillePredefinie (
\begin{DoxyParamCaption}
\item[{{\bf Plateau} $\ast$}]{ pPlateau}
\end{DoxyParamCaption}
)}}
\label{plateau_8c_aa2bb750ea8570ed32eb894c7249d2e35}


Rempli le plateau avec une grille prédéfinie de taille 4$\ast$4 E D R C A N V C I R Q A E B R U. 


\begin{DoxyParams}{Paramètres}
\item[{\em pPlateau}]Le plateau à remplire \end{DoxyParams}

\hypertarget{resolveur_8c}{\section{Référence du fichier resolveur.\-c}
\label{resolveur_8c}\index{resolveur.\-c@{resolveur.\-c}}
}


Gestion du résolveur Permet de résoudre une grille de Boggle, une fois résolue, la solution se trouve dans le module \hyperlink{structSolution}{Solution}.  


{\ttfamily \#include $<$string.\-h$>$}\\*
{\ttfamily \#include \char`\"{}dictionnaire.\-h\char`\"{}}\\*
{\ttfamily \#include \char`\"{}resolveur.\-h\char`\"{}}\\*
{\ttfamily \#include \char`\"{}solution.\-h\char`\"{}}\\*
\subsection*{Fonctions}
\begin{DoxyCompactItemize}
\item 
void \hyperlink{resolveur_8c_a1eacf34755bde03a6bc60c79383bd657}{recurse} (\hyperlink{structPlateau}{Plateau} $\ast$p\-Plateau, int x, int y, int depth, char $\ast$choices, \hyperlink{structDico}{Dico} p\-Dico)
\begin{DoxyCompactList}\small\item\em Fonction privée récursive permettant de résoudre une grille de Boggle. \end{DoxyCompactList}\item 
void \hyperlink{resolveur_8c_a8696ca50e0d0e6d4c05399c26372709d}{resolveur} (\hyperlink{structPlateau}{Plateau} $\ast$p\-Plateau, \hyperlink{structDico}{Dico} p\-Dico)
\begin{DoxyCompactList}\small\item\em Résoud une grille de boggle. \end{DoxyCompactList}\end{DoxyCompactItemize}


\subsection{Description détaillée}
Gestion du résolveur Permet de résoudre une grille de Boggle, une fois résolue, la solution se trouve dans le module \hyperlink{structSolution}{Solution}. 

\subsection{Documentation des fonctions}
\hypertarget{resolveur_8c_a1eacf34755bde03a6bc60c79383bd657}{\index{resolveur.\-c@{resolveur.\-c}!recurse@{recurse}}
\index{recurse@{recurse}!resolveur.c@{resolveur.\-c}}
\subsubsection[{recurse}]{\setlength{\rightskip}{0pt plus 5cm}void recurse (
\begin{DoxyParamCaption}
\item[{{\bf Plateau} $\ast$}]{p\-Plateau, }
\item[{int}]{x, }
\item[{int}]{y, }
\item[{int}]{depth, }
\item[{char $\ast$}]{choices, }
\item[{{\bf Dico}}]{p\-Dico}
\end{DoxyParamCaption}
)}}\label{resolveur_8c_a1eacf34755bde03a6bc60c79383bd657}


Fonction privée récursive permettant de résoudre une grille de Boggle. 


\begin{DoxyParams}{Paramètres}
{\em p\-Plateau} & Le lateau à résoudre \\
\hline
{\em x} & L'abscisse de la lettre pour laquelle on part \\
\hline
{\em y} & L'ordonné de la lettre pour laquelle on part \\
\hline
{\em depth} & La profondeur à laquelle on se trouve \\
\hline
{\em choices} & Le mot formé actuellement \\
\hline
{\em p\-Dico} & Le dictionnaire dans laquel chercher les mots \\
\hline
\end{DoxyParams}
\hypertarget{resolveur_8c_a8696ca50e0d0e6d4c05399c26372709d}{\index{resolveur.\-c@{resolveur.\-c}!resolveur@{resolveur}}
\index{resolveur@{resolveur}!resolveur.c@{resolveur.\-c}}
\subsubsection[{resolveur}]{\setlength{\rightskip}{0pt plus 5cm}void resolveur (
\begin{DoxyParamCaption}
\item[{{\bf Plateau} $\ast$}]{p\-Plateau, }
\item[{{\bf Dico}}]{p\-Dico}
\end{DoxyParamCaption}
)}}\label{resolveur_8c_a8696ca50e0d0e6d4c05399c26372709d}


Résoud une grille de boggle. 


\begin{DoxyParams}{Paramètres}
{\em p\-Plateau} & Le plateau à résoudre \\
\hline
{\em choices} & \\
\hline
{\em p\-Dico} & le dictionnaire \\
\hline
\end{DoxyParams}

\hypertarget{util_8c}{\section{Référence du fichier util.\-c}
\label{util_8c}\index{util.\-c@{util.\-c}}
}


Fonctions utiles à l'ensemble du projet Toutes les fonctions de bases utiles au projet, ces fonctions travaillent sur des types de bae et ne sont pas spécfiques au projet, ce module permet de meux organiser le code.  


{\ttfamily \#include $<$string.\-h$>$}\\*
{\ttfamily \#include $<$wchar.\-h$>$}\\*
{\ttfamily \#include $<$stdlib.\-h$>$}\\*
{\ttfamily \#include $<$stdbool.\-h$>$}\\*
{\ttfamily \#include \char`\"{}util.\-h\char`\"{}}\\*
{\ttfamily \#include \char`\"{}jeu.\-h\char`\"{}}\\*
{\ttfamily \#include \char`\"{}case.\-h\char`\"{}}\\*
\subsection*{Fonctions}
\begin{DoxyCompactItemize}
\item 
void \hyperlink{util_8c_a3c1c39d9849edbfc6030dad294f803c2}{util\-\_\-affichage\-Debug} (const char $\ast$p\-Nom\-Fonction, const char $\ast$p\-Chaine)
\begin{DoxyCompactList}\small\item\em Affiche une chaine de caractère sur la console uniquement si le define M\-O\-D\-E\-\_\-\-D\-E\-B\-U\-G est vrai. \end{DoxyCompactList}\item 
\hypertarget{util_8c_ac2c251fd00334d5987228dd6f5431f02}{void \hyperlink{util_8c_ac2c251fd00334d5987228dd6f5431f02}{util\-\_\-afficher\-Table\-Ascii} (void)}\label{util_8c_ac2c251fd00334d5987228dd6f5431f02}

\begin{DoxyCompactList}\small\item\em Affiche la tableau A\-S\-C\-I\-I. \end{DoxyCompactList}\item 
char $\ast$ \hyperlink{util_8c_a0e84020f177fa04e8894bf7f1d159442}{util\-\_\-supprimer\-Accents} (const char $\ast$p\-Chaine)
\begin{DoxyCompactList}\small\item\em Supprime les accents d'une chaine de caratère. \end{DoxyCompactList}\item 
void \hyperlink{util_8c_a62c93d0b64c8cb04e51e6a116976f741}{util\-\_\-uppercase} (char $\ast$p\-Chaine)
\begin{DoxyCompactList}\small\item\em Modifie la chaine de caractère afin qu'elle soit en majuscule. \end{DoxyCompactList}\item 
char \hyperlink{util_8c_a731fef986576fd9259a24619ca099bb2}{util\-\_\-nb\-Aleatoire} (const char p\-Debut, const char p\-Fin)
\begin{DoxyCompactList}\small\item\em Retourne un nombre aléatoire entre p\-Debut et p\-Fin. \end{DoxyCompactList}\item 
int \hyperlink{util_8c_a08242f964f0e2453993b26df48953c65}{util\-\_\-substr} (const char $\ast$chaine, int debut, int fin, char $\ast$result)
\begin{DoxyCompactList}\small\item\em Créer une sous-\/chaine de caractère de chaine depuis début jusqu'à fin. \end{DoxyCompactList}\item 
void \hyperlink{util_8c_a76c1cbcd1bd43ad75cb93a6ee17bd65e}{util\-\_\-echanger} (char $\ast$$\ast$tableau, int a, int b)
\begin{DoxyCompactList}\small\item\em Echange deux variables dans un tableau de chaine de caractères. \end{DoxyCompactList}\item 
void \hyperlink{util_8c_a2bffc43ebb03a4eb18361ac919369aba}{util\-\_\-quick\-Sort} (char $\ast$$\ast$tableau, int debut, int fin)
\begin{DoxyCompactList}\small\item\em Effectue un tri sur une porition d'un tableau de chaine de caractère s. \end{DoxyCompactList}\item 
void \hyperlink{util_8c_a61ce43583b52fd8f8b69b34dcb411ef9}{util\-\_\-deplacer\-Curseur\-Dun\-Mot} (F\-I\-L\-E $\ast$p\-Fichier, const int p\-Sens)
\begin{DoxyCompactList}\small\item\em Déplace le curseurdans le fichier au début du mot précédent ou suivant en fonction de p\-Sens. \end{DoxyCompactList}\item 
\-\_\-\-Bool \hyperlink{util_8c_a0311f1c6b9a4bc8506c8490c63c6bf6b}{util\-\_\-is\-In\-Array} (const \hyperlink{structCase}{Case} $\ast$p\-Tableau, const int p\-Taille, const \hyperlink{structCase}{Case} p\-Case)
\begin{DoxyCompactList}\small\item\em Retourne vrai si la case p\-Case est présente dans le tableau. \end{DoxyCompactList}\item 
void \hyperlink{util_8c_ae959a017ee969585d79f68fffabf710d}{util\-\_\-conversion\-Temps} (const time\-\_\-t p\-Timestamp, int $\ast$p\-Minutes, int $\ast$p\-Secondes)
\begin{DoxyCompactList}\small\item\em Convertis un nombre de secondes en minutes et secondes. \end{DoxyCompactList}\end{DoxyCompactItemize}


\subsection{Description détaillée}
Fonctions utiles à l'ensemble du projet Toutes les fonctions de bases utiles au projet, ces fonctions travaillent sur des types de bae et ne sont pas spécfiques au projet, ce module permet de meux organiser le code. 

\subsection{Documentation des fonctions}
\hypertarget{util_8c_a3c1c39d9849edbfc6030dad294f803c2}{\index{util.\-c@{util.\-c}!util\-\_\-affichage\-Debug@{util\-\_\-affichage\-Debug}}
\index{util\-\_\-affichage\-Debug@{util\-\_\-affichage\-Debug}!util.c@{util.\-c}}
\subsubsection[{util\-\_\-affichage\-Debug}]{\setlength{\rightskip}{0pt plus 5cm}void util\-\_\-affichage\-Debug (
\begin{DoxyParamCaption}
\item[{const char $\ast$}]{p\-Nom\-Fonction, }
\item[{const char $\ast$}]{p\-Chaine}
\end{DoxyParamCaption}
)\hspace{0.3cm}{\ttfamily [inline]}}}\label{util_8c_a3c1c39d9849edbfc6030dad294f803c2}


Affiche une chaine de caractère sur la console uniquement si le define M\-O\-D\-E\-\_\-\-D\-E\-B\-U\-G est vrai. 


\begin{DoxyParams}{Paramètres}
{\em p\-Nom\-Fonction} & Le nom de la fonction de laquelle est appellée cette fonction \\
\hline
{\em p\-Chaine} & La chaine de caractère à afficher \\
\hline
\end{DoxyParams}
\hypertarget{util_8c_ae959a017ee969585d79f68fffabf710d}{\index{util.\-c@{util.\-c}!util\-\_\-conversion\-Temps@{util\-\_\-conversion\-Temps}}
\index{util\-\_\-conversion\-Temps@{util\-\_\-conversion\-Temps}!util.c@{util.\-c}}
\subsubsection[{util\-\_\-conversion\-Temps}]{\setlength{\rightskip}{0pt plus 5cm}void util\-\_\-conversion\-Temps (
\begin{DoxyParamCaption}
\item[{const time\-\_\-t}]{p\-Timestamp, }
\item[{int $\ast$}]{p\-Minutes, }
\item[{int $\ast$}]{p\-Secondes}
\end{DoxyParamCaption}
)}}\label{util_8c_ae959a017ee969585d79f68fffabf710d}


Convertis un nombre de secondes en minutes et secondes. 


\begin{DoxyParams}{Paramètres}
{\em p\-Timestamp} & Le nombre de secondes à converti \\
\hline
{\em p\-Minutes} & Le nombre de minutes résultat \\
\hline
{\em p\-Secondes} & Le nombre de secondes résultat \\
\hline
\end{DoxyParams}
\hypertarget{util_8c_a61ce43583b52fd8f8b69b34dcb411ef9}{\index{util.\-c@{util.\-c}!util\-\_\-deplacer\-Curseur\-Dun\-Mot@{util\-\_\-deplacer\-Curseur\-Dun\-Mot}}
\index{util\-\_\-deplacer\-Curseur\-Dun\-Mot@{util\-\_\-deplacer\-Curseur\-Dun\-Mot}!util.c@{util.\-c}}
\subsubsection[{util\-\_\-deplacer\-Curseur\-Dun\-Mot}]{\setlength{\rightskip}{0pt plus 5cm}void util\-\_\-deplacer\-Curseur\-Dun\-Mot (
\begin{DoxyParamCaption}
\item[{F\-I\-L\-E $\ast$}]{p\-Fichier, }
\item[{const int}]{p\-Sens}
\end{DoxyParamCaption}
)}}\label{util_8c_a61ce43583b52fd8f8b69b34dcb411ef9}


Déplace le curseurdans le fichier au début du mot précédent ou suivant en fonction de p\-Sens. 


\begin{DoxyParams}{Paramètres}
{\em p\-Fichier} & Le fichier surlequelle s'apppliqeu le curseur. Celui-\/ci doit être ouvert en lecture \\
\hline
{\em p\-Sens} & le sens du déplacement M\-O\-T\-\_\-\-P\-R\-E\-C\-E\-D\-E\-N\-T ou M\-O\-T\-\_\-\-S\-U\-I\-V\-A\-N\-T \\
\hline
\end{DoxyParams}
\hypertarget{util_8c_a76c1cbcd1bd43ad75cb93a6ee17bd65e}{\index{util.\-c@{util.\-c}!util\-\_\-echanger@{util\-\_\-echanger}}
\index{util\-\_\-echanger@{util\-\_\-echanger}!util.c@{util.\-c}}
\subsubsection[{util\-\_\-echanger}]{\setlength{\rightskip}{0pt plus 5cm}void util\-\_\-echanger (
\begin{DoxyParamCaption}
\item[{char $\ast$$\ast$}]{tableau, }
\item[{int}]{a, }
\item[{int}]{b}
\end{DoxyParamCaption}
)}}\label{util_8c_a76c1cbcd1bd43ad75cb93a6ee17bd65e}


Echange deux variables dans un tableau de chaine de caractères. 


\begin{DoxyParams}{Paramètres}
{\em tableau} & Le tableau sur laquelle on effectue l'échange \\
\hline
{\em a} & La première valeure à échanger \\
\hline
{\em b} & La seconde valeure \\
\hline
\end{DoxyParams}
\hypertarget{util_8c_a0311f1c6b9a4bc8506c8490c63c6bf6b}{\index{util.\-c@{util.\-c}!util\-\_\-is\-In\-Array@{util\-\_\-is\-In\-Array}}
\index{util\-\_\-is\-In\-Array@{util\-\_\-is\-In\-Array}!util.c@{util.\-c}}
\subsubsection[{util\-\_\-is\-In\-Array}]{\setlength{\rightskip}{0pt plus 5cm}\-\_\-\-Bool util\-\_\-is\-In\-Array (
\begin{DoxyParamCaption}
\item[{const {\bf Case} $\ast$}]{p\-Tableau, }
\item[{const int}]{p\-Taille, }
\item[{const {\bf Case}}]{p\-Case}
\end{DoxyParamCaption}
)}}\label{util_8c_a0311f1c6b9a4bc8506c8490c63c6bf6b}


Retourne vrai si la case p\-Case est présente dans le tableau. 


\begin{DoxyParams}{Paramètres}
{\em p\-Tableau} & Le tableau sur lequele s'effectue le test \\
\hline
{\em p\-Taille} & La taille du tableau \\
\hline
{\em p\-Case} & La case à chercher \\
\hline
\end{DoxyParams}
\begin{DoxyReturn}{Renvoie}
Vrai si la case existe dans le tableau. 
\end{DoxyReturn}
\hypertarget{util_8c_a731fef986576fd9259a24619ca099bb2}{\index{util.\-c@{util.\-c}!util\-\_\-nb\-Aleatoire@{util\-\_\-nb\-Aleatoire}}
\index{util\-\_\-nb\-Aleatoire@{util\-\_\-nb\-Aleatoire}!util.c@{util.\-c}}
\subsubsection[{util\-\_\-nb\-Aleatoire}]{\setlength{\rightskip}{0pt plus 5cm}char util\-\_\-nb\-Aleatoire (
\begin{DoxyParamCaption}
\item[{const char}]{p\-Debut, }
\item[{const char}]{p\-Fin}
\end{DoxyParamCaption}
)}}\label{util_8c_a731fef986576fd9259a24619ca099bb2}


Retourne un nombre aléatoire entre p\-Debut et p\-Fin. 


\begin{DoxyParams}{Paramètres}
{\em p\-Debut} & la borne inférieur \\
\hline
{\em p\-Fin} & la borne supérieur \\
\hline
\end{DoxyParams}
\begin{DoxyReturn}{Renvoie}
Le nombre généré 
\end{DoxyReturn}
\hypertarget{util_8c_a2bffc43ebb03a4eb18361ac919369aba}{\index{util.\-c@{util.\-c}!util\-\_\-quick\-Sort@{util\-\_\-quick\-Sort}}
\index{util\-\_\-quick\-Sort@{util\-\_\-quick\-Sort}!util.c@{util.\-c}}
\subsubsection[{util\-\_\-quick\-Sort}]{\setlength{\rightskip}{0pt plus 5cm}void util\-\_\-quick\-Sort (
\begin{DoxyParamCaption}
\item[{char $\ast$$\ast$}]{tableau, }
\item[{int}]{debut, }
\item[{int}]{fin}
\end{DoxyParamCaption}
)}}\label{util_8c_a2bffc43ebb03a4eb18361ac919369aba}


Effectue un tri sur une porition d'un tableau de chaine de caractère s. 

Ce tri utilise l'algorithme du tri rapide. 
\begin{DoxyParams}{Paramètres}
{\em tableau} & Le tableau à trier \\
\hline
{\em debut} & Le début du tableau à trier \\
\hline
{\em fin} & La fin du tableau à trier \\
\hline
\end{DoxyParams}
\hypertarget{util_8c_a08242f964f0e2453993b26df48953c65}{\index{util.\-c@{util.\-c}!util\-\_\-substr@{util\-\_\-substr}}
\index{util\-\_\-substr@{util\-\_\-substr}!util.c@{util.\-c}}
\subsubsection[{util\-\_\-substr}]{\setlength{\rightskip}{0pt plus 5cm}int util\-\_\-substr (
\begin{DoxyParamCaption}
\item[{const char $\ast$}]{chaine, }
\item[{int}]{debut, }
\item[{int}]{fin, }
\item[{char $\ast$}]{result}
\end{DoxyParamCaption}
)}}\label{util_8c_a08242f964f0e2453993b26df48953c65}


Créer une sous-\/chaine de caractère de chaine depuis début jusqu'à fin. 


\begin{DoxyParams}{Paramètres}
{\em chaine} & La chaine sur laquelle s'applique la fonction \\
\hline
{\em debut} & Le début du découpage \\
\hline
{\em fin} & La fin du découpage \\
\hline
{\em result} & La chaine résultat \\
\hline
\end{DoxyParams}
\begin{DoxyReturn}{Renvoie}
La taille de la chaine résultat 
\end{DoxyReturn}
\hypertarget{util_8c_a0e84020f177fa04e8894bf7f1d159442}{\index{util.\-c@{util.\-c}!util\-\_\-supprimer\-Accents@{util\-\_\-supprimer\-Accents}}
\index{util\-\_\-supprimer\-Accents@{util\-\_\-supprimer\-Accents}!util.c@{util.\-c}}
\subsubsection[{util\-\_\-supprimer\-Accents}]{\setlength{\rightskip}{0pt plus 5cm}char$\ast$ util\-\_\-supprimer\-Accents (
\begin{DoxyParamCaption}
\item[{const char $\ast$}]{p\-Chaine}
\end{DoxyParamCaption}
)}}\label{util_8c_a0e84020f177fa04e8894bf7f1d159442}


Supprime les accents d'une chaine de caratère. 


\begin{DoxyParams}{Paramètres}
{\em p\-Chaine} & La chaine de caractère avec les accents \\
\hline
\end{DoxyParams}
\begin{DoxyReturn}{Renvoie}
La nouvelle chaine de caractère sans accents 
\end{DoxyReturn}
\hypertarget{util_8c_a62c93d0b64c8cb04e51e6a116976f741}{\index{util.\-c@{util.\-c}!util\-\_\-uppercase@{util\-\_\-uppercase}}
\index{util\-\_\-uppercase@{util\-\_\-uppercase}!util.c@{util.\-c}}
\subsubsection[{util\-\_\-uppercase}]{\setlength{\rightskip}{0pt plus 5cm}void util\-\_\-uppercase (
\begin{DoxyParamCaption}
\item[{char $\ast$}]{p\-Chaine}
\end{DoxyParamCaption}
)}}\label{util_8c_a62c93d0b64c8cb04e51e6a116976f741}


Modifie la chaine de caractère afin qu'elle soit en majuscule. 


\begin{DoxyParams}{Paramètres}
{\em p\-Chaine} & La chaine de caractère à modifier \\
\hline
\end{DoxyParams}

\printindex
\end{document}
