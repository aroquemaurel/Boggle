\documentclass[12pt,a4paper,openany]{article}
\usepackage{lmodern}
\usepackage[table]{xcolor}
\input{../includesLaTeX/couleurs.tex}

\usepackage[utf8]{inputenc} \usepackage[T1]{fontenc}
\usepackage[francais]{babel}
\usepackage[top=1.7cm, bottom=1.7cm, left=1.7cm, right=1.7cm]{geometry}
\usepackage{verbatim}
\usepackage[urlbordercolor={1 1 1}, linkbordercolor={1 1 1}, linkcolor=vert1, urlcolor=bleu, colorlinks=true]{hyperref}
\usepackage{tikz} %Vectoriel
\usepackage{listings}
\usepackage{fancyhdr}
\usepackage{multido}
\usepackage{float}
\usepackage{amssymb}

\newcommand{\titre}{Dossier de conception préliminaire}

\newcommand{\pole}{}
\newcommand{\sigle}{}

\newcommand{\semestre}{4}

\input{../includesLaTeX/listings.tex}
\input{../includesLaTeX/l2/docsProjet2.tex}
\input{../includesLaTeX/remarquesExempleAttention.tex}
\input{../includesLaTeX/polices.tex}
\makeatletter
\def\thickhrulefill{\leavevmode \leaders \hrule height 1ex \hfill \kern \z@}

\newlength{\sectiontitleindent}
\newlength{\subsectiontitleindent}
\newlength{\subsubsectiontitleindent}
\setlength{\sectiontitleindent}{-1cm}
\setlength{\subsectiontitleindent}{-.5cm}
\setlength{\subsubsectiontitleindent}{-.25cm}

\renewcommand{\section}{%
	\@startsection%
	{section}%
	{1}%
	{\sectiontitleindent}%
	{-3.5ex plus -1ex minus -.2ex}%
	{2.3ex plus.2ex}%
	{\sectionfont\Large}
}
\renewcommand{\subsection}{%
	\@startsection%
	{subsection}%
	{2}%
	{\subsectiontitleindent}%
	{-3.5ex plus -1ex minus -.2ex}%
	{2.3ex plus.2ex}%
	{\sectionfont\large}
}

\renewcommand{\subsubsection}{%
	\@startsection%
	{subsubsection}%
	{3}%
	{\subsubsectiontitleindent}%
	{-3.5ex plus -1ex minus -.2ex}%
	{2.3ex plus.2ex}%
	{\sectionfont\normalsize}
}

\makeatother

\newcommand{\lien}[1]{
 $\vartriangleright$ \url{#1}
 }

\newcommand{\pfp}{\texttt{pfp}}

\newcommand{\ifp}{\texttt{if}}
\newcommand{\elsep}{\texttt{else}}

\makeatother
\includeonly {
}
\begin{document}
	\setcounter{tocdepth}{2}
	\setcounter{secnumdepth}{3}
	\maketitle
	\tableofcontents
	\newpage
	\section{But du document}
	C'est une description de haut niveau du produit, c'est-à-dire l'architecture générale du système, en termes de << modules >>, de sous modules et de leurs
	interactions. De plus, chaque module doit être décrit (définition des interfaces et des fonctionnalités générales). Ce document doit en premier lieu asseoir
	la confiance en la finalité et la faisabilité du produit, et, en second lieu, servir de base pour l'estimation des tâches à effectuer et du calendrier de
	leur réalisation.

	Le << Dossier de Conception Préliminaire >> doit également mettre en évidence le plan de tests, en termes de besoins de l'utilisateur, et montrer que l'on peut
	y satisfaire grâce à l'architecture proposée.
	\section{Diagramme de décomposition en modules}
	% TODO
	\section{Description des différents modules}
		\subsection{Module \texttt{Plateau}}
		\begin{table}[H]
		\rowcolors{1}{grisgris}{}
			\centering
		\begin{tabular}{p{5cm} p{12cm}}
			\textbf{Rôle} & Gérer la grille de Boggle\\
			\textbf{Type de données} & Tableau à deux dimensions de \texttt{char} \\
			\textbf{Dépendances} & Aucune \\
			\textbf{Fonctionnalités fournies} & Générer la grille, Retourner la lettre concernant une case donnée\\
		\end{tabular}
		\caption{Module \texttt{Plateau}}
	\end{table}
		\subsection{Module \texttt{Dictionnaire}}
		\begin{tabular}{p{5cm} p{12cm}}
			\textbf{Rôle} & Gérer le dictionnaire du Boggle\\
			\textbf{Type de données} & Fichier \texttt{FILE*} pointant sur le dictionnaire\\
			\textbf{Dépendances} & Aucune\\
			\textbf{Fonctionnalités fournies} & Ouvrir le dictionnaire, parcourir le dictionnaire, dire si un mot est présent dans le dictionnaire ou non. 
		\end{tabular}
		\subsection{Module \texttt{Résolveur}}
		\begin{tabular}{p{5cm} p{12cm}}
			\textbf{Rôle} & Résoudre une grille de Boggle\\
			\textbf{Type de données} & Structure contenant la grille de Boggle, le dictionnaire et un tableau de \texttt{char*} avec tous les mots possibles\\
			\textbf{Dépendances} & \texttt{Dictionnaire}, \texttt{Plateau}\\
			\textbf{Fonctionnalités fournies} & Résoudre la grille, signaler si un mot est présent dans la grille, retourner la liste des mots de la grille
			commençant par une lettre. 
		\end{tabular}
		\subsection{Module \texttt{Jeu}}
		\begin{tabular}{p{5cm} p{12cm}}
			\textbf{Rôle} & Jouer au Boggle \\ 
			\textbf{Type de données} & \\ 
			\textbf{Dépendances} & Plateau, InterfaceGraphique, InterfaceTexte, Résolveur\\
			\textbf{Fonctionnalités fournies} & Proposer une lettre, Lancer le compte à rebours, SIgnaler si un mot proposé est correct, retourner le nombre de
			point obtenus, signaler si le joueur à gagner le jeu ou non
		\end{tabular}
		\subsection{Module \texttt{Interface}}
		\subsection{Module \texttt{Utile}}
	\section{Répartition des tâches entre chaque membre du groupe}
		\begin{table}[H]
		\rowcolors{2}{grisgris}{}
			\centering
		\begin{tabular}{m{12cm} m{5cm}}
			\rowcolor{gris}\textbf{Tâche} & \textbf{Membre}\\
			..& Fabrice \\
			..& Antoine\\
			..& Antoine\\
		\end{tabular}
		\caption{Répartition des tâches}
	\end{table}
	\section{Calendrier de réalisation des tâches}
	\section{Plan de tests}
	
	
\end{document}

