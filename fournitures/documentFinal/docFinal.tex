\documentclass[12pt,a4paper,openany]{article}
\usepackage{lmodern}
\usepackage[table]{xcolor}
\input{../includesLaTeX/couleurs.tex}

\usepackage[utf8]{inputenc} \usepackage[T1]{fontenc}
\usepackage[francais]{babel}
\usepackage[top=1.7cm, bottom=1.7cm, left=1.7cm, right=1.7cm]{geometry}
\usepackage{verbatim}
\usepackage[urlbordercolor={1 1 1}, linkbordercolor={1 1 1}, linkcolor=vert1, urlcolor=bleu, colorlinks=true]{hyperref}
\usepackage{tikz} %Vectoriel
\usepackage{listings}
\usepackage{fancyhdr}
\usepackage{multido}
\usepackage{float}
\usepackage{amssymb}

\newcommand{\titre}{Dossier }% TODO

\newcommand{\pole}{}
\newcommand{\sigle}{}

\newcommand{\semestre}{4}

\input{../includesLaTeX/listings.tex}
\input{../includesLaTeX/l2/docsProjet2.tex}
\input{../includesLaTeX/remarquesExempleAttention.tex}
\input{../includesLaTeX/polices.tex}
\makeatletter
\def\thickhrulefill{\leavevmode \leaders \hrule height 1ex \hfill \kern \z@}

\newlength{\sectiontitleindent}
\newlength{\subsectiontitleindent}
\newlength{\subsubsectiontitleindent}
\setlength{\sectiontitleindent}{-1cm}
\setlength{\subsectiontitleindent}{-.5cm}
\setlength{\subsubsectiontitleindent}{-.25cm}

\renewcommand{\section}{%
	\@startsection%
	{section}%
	{1}%
	{\sectiontitleindent}%
	{-3.5ex plus -1ex minus -.2ex}%
	{2.3ex plus.2ex}%
	{\sectionfont\Large}
}
\renewcommand{\subsection}{%
	\@startsection%
	{subsection}%
	{2}%
	{\subsectiontitleindent}%
	{-3.5ex plus -1ex minus -.2ex}%
	{2.3ex plus.2ex}%
	{\sectionfont\large}
}

\renewcommand{\subsubsection}{%
	\@startsection%
	{subsubsection}%
	{3}%
	{\subsubsectiontitleindent}%
	{-3.5ex plus -1ex minus -.2ex}%
	{2.3ex plus.2ex}%
	{\sectionfont\normalsize}
}

\makeatother

\newcommand{\lien}[1]{
 $\vartriangleright$ \url{#1}
 }

\newcommand{\pfp}{\texttt{pfp}}

\newcommand{\ifp}{\texttt{if}}
\newcommand{\elsep}{\texttt{else}}

\makeatother
\includeonly {
}
\begin{document}
	\setcounter{tocdepth}{2}
	\setcounter{secnumdepth}{3}
	\maketitle
	\tableofcontents
	\newpage
	\section{But du document}
	C'est une description de haut niveau du produit, c'est-à-dire l'architecture générale du système, en termes de << modules >>, de sous modules et de leurs
	interactions. De plus, chaque module doit être décrit (définition des interfaces et des fonctionnalités générales). Ce document doit en premier lieu asseoir
	la confiance en la finalité et la faisabilité du produit, et, en second lieu, servir de base pour l'estimation des tâches à effectuer et du calendrier de
	leur réalisation.

	Le << Dossier de Conception Préliminaire >> doit également mettre en évidence le plan de tests, en termes de besoins de l'utilisateur, et montrer que l'on peut
	y satisfaire grâce à l'architecture proposée.
	\section{Compilation}
	La compilation du projet se fait à l'aide de l'utilitaire \texttt{Make}, ainsi la simple
	commande \texttt{make} à la racine du projet suffit à compiler le projet.

	Cependant, afin de pouvoir compiler le projet, il est indispensable de posséder la
	bibliothèque \textit{Ncurses} sur sa machine, sinon la compilation ne fonctionnerai pas.\\ 
	Il est possible de l'installer avec la commande \texttt{apt-get install libncurses5-dev}
	sur les Linux utilisant le gestionnaire de paquet de Debian.

	Les tests quant à eux se compile à l'aide de la commande \texttt{make test}, cette
	commande va compiler puis exécuter tous les tests, cependant si vous ne possédez pas
	\textit{CUnit} sur votre machine, il est également indispensable de taper la commande
	suivante afin de signaler au système l'emplacement de la bibliothèque.

	\begin{lstlisting}[language=sh,numbers=none]
LD_LIBRARY_PATH=$\$$LD_LIBRARY_PATH:`pwd`/Cunit/lib && export LD_LIBRARY_PATH		
	\end{lstlisting}

	Cette commande doit être employée à chaque ouverture d'une nouvelle console, la variable étant attachée à une seule console.  

	\section{Exécution}
	Afin de tester l'application, deux types de tests sont présent dans le projet.

		\subsection{Les tests fonctionnels} Ces tests sont tous présents dans l'executable
			\texttt{./boggle}. Afin d'appeler les différentes fonctionnalités du programme, il
			est nécessaire de faire passer un paramètre, celui-ci peut prendre la forme d'une
			de ces trois chaines de caractères : 
			\begin{description}
				\item[\texttt{--solveur}] Correspond à la version 1 du projet.
Afin d'appeler la version 1 de l'application, l'exécutable doit être appelé à l'aide de
l'argument \texttt{--solveur}

Dans cette version, une grille carrée de la taille demandée par l'utilisateur est génére, en
tenant compte de la fréquence des lettres dans la langue Française. Une fois la grille
générée, la position d'une case est demandée à l'utilisateur, l'utilisateur entre donc les
deux coordonnées, et tous les mots commençant par cette case seront affichés à l'écran.

Attention, les coordonnées de la grille commences à zéro.
				\item[\texttt{--texte}]
					Afin d'appeler la version 2, l'exécutable doit être appelé à l'aide de
					l'argument \texttt{--text}

					Cette version fait appel à la version 1, en effet, au lancement de
					l'application, il est de nouveau demandé la taille de la grille, ensuite
					l'intégralité de la grille générer est résolue. Une fois cette étape
					franchie, l'utilisateur à 3 minutes pour entrer le plus de mots possibles,
					l'application lui signalant si le mot est accepté ou non, une fois ce
					temps impartis, la solution est affichée, puis le nombre de points obtenu
					par le joueur.
				\item[\texttt{--ncurses}]
					Afin d'appeler la version 3, l'exécutable doit être appelé à l'aide de
					l'argument \texttt{--ncurses}

					Cette version suit le même principe que la version précédente, à la
					différence près qu'elle utilise la bibliothèque \textit{Ncurses}. Ainsi, la saisie
					des mots se fait dorénavant avec les touches fléchées du clavier, et
					espace pour ajouter une lettre au mot. Pour proposer le mot surligné, la
					touche entrée doit être appuyée. Il est également possible de demander le
					nombre de mots commençant par la case sélectionnée à l'aide de la touche
					h.

					Une fois les 3 minutes écoulées, les mots proposés par l'utilisateur et le
					nombre de points obtenus sont affichés, il est proposé à l'utilisateur
					d'afficher la solution complète. 
			\end{description}
		\subsection{Les tests unitaires}
	\section{Documentation}
	\section{Gestion de projet}
	\subsection{Conventions de codage}
	\subsection{Logiciel d'analyse de code: Sonar}
	\subsection{Logiciel de gestion de projet: Redmine}
	\subsection{Logiciel de versionnement: Git}

\end{document}

