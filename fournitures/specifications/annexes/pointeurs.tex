\chapter{Cours sur les pointeurs en C}\label{pointeurs}
Déjà vu par le passages de paramètres, nous allons donc ici commencer par un rappel, et nous verrons l'allocation dynamique de mémoire.
\section{Syntaxe}
	\subsection{Déclaration}
\begin{lstlisting}[language=C, numbers=none,frame=none, caption=Syntaxe de déclaration d'un pointeur]
typePointé* nomPointeur
\end{lstlisting}
\lstinputlisting[caption=Exemple de déclaration, numbers=none,language=C]{annexes/ptr1.c}
	\subsection{Utilisation}
\begin{lstlisting}[language=C, numbers=none,frame=none, caption=Syntaxe utilisation d'un pointeur]
nomPointeur // manipule l'adresse
*nomPointeur //manipule la zone pointée
\end{lstlisting}
\begin{lstlisting}[language=C, numbers=none,frame=none, caption=Exemple d'utilisation d'un pointeur]
pe=&n; //opérateur d'adressage
\end{lstlisting}

\subsection{Constante}
	\texttt{NULL} représente une adresse inexistante.
	\begin{lstlisting}[language=C, numbers=none,frame=none, caption=Exemple d'utilisation de la constante \texttt{NULL}]
pe = NULL;
*pe; // Erreur à l'exécution
	\end{lstlisting}

\section{Opérateur autorisés sur les pointeurs}
\subsection{L'affectation}
\begin{verbatim}
nomPointeur =  expression correspondant à une adresse ou à NULL
\end{verbatim}

\subsection{Addition et la soustraction entre un pointeur et un entier}
\begin{verbatim}
nomPointeur = nomPointeur + 10; 
nomPointeur = nomPointeur - 15; 
\end{verbatim}
On obtient une expression correspondant à une adresse
\begin{verbatim}
pe = pe+10; //pe contient l'adresse du 10e entier après la valeu initalie de pe.
\end{verbatim}
\attention{À utiliser que si pe pointe sur un tableau}

\subsection{Soustraction de deux pointeurs}
Renvoi un entier donnant le nombre d'éléments pointés entre les deux pointeurs
\attention{Uniquement si les deux pointeurs sont sur le même tableau}

\subsection{Comparaison sur des pointeurs}
Ce sont les opérateurs de comparaison classique : \texttt{= =} et \texttt{!=}

\subsection{Allocation dynamique de mémoire}
\begin{lstlisting}[language=C, numbers=none,frame=none, caption=Syntaxe d'allocation dynamique]
nomPointeur = (typePointeur) malloc(sizeof(typePointé));
nomPointeur = (typePointé*) malloc(n*sizeof(typePointé));
\end{lstlisting}
\begin{lstlisting}[language=C, numbers=none,frame=none, caption=Exemple d'allocation dynamique]
	int *e;
	pe = (int*) malloc(sizeof(int));
\end{lstlisting}
\begin{enumerate}
	\item Le programme demande au gestionnaire mémoire à avoir une place de la taille \texttt{sizeof(int)}
	\item Si la place est disponible retourne l'adresse demandée ou la première case du <<tableau>> dynamique
	\item Sinon retourne \texttt{NULL}
\end{enumerate}

\subsection{Libération dynamique de mémoire}
\begin{lstlisting}[language=C, numbers=none,frame=none, caption=Syntaxe de libération de mémoire]
free(nomPointeur);
\end{lstlisting}
\begin{enumerate}
	\item Le programme contact le gestionnaire mémoire 
	\item Le gestionnaire mémoire <<libère>> la place
\end{enumerate}

Cela veut dire que la place n'est plus réservé au programme, elle pourra être alloué à un autre programme.
\attention{Le gestionnaire de mémoire ne met pas à jour la case mémoire, celle-ci contient toujours la valeur, si personne ne récupère la case, il sera toujours
possible d'accéder à la donnée. C'est donc aléatoire, c'est une source d'erreurs.}

\section{Pointeur sur fonction}
Un pointeur de fonction est un pointeur qui contient l'adresse d'une fonction.

\subsection{Syntaxe}
\begin{lstlisting}[language=C, numbers=none,caption=Déclaration d'un pointeur de fonction]
typederetour (*nomPtrFonction)(liteDesParametres type1 p1, type2 p2, type3 p3);
\end{lstlisting}

\subsection{Utilisation}
\begin{lstlisting}[language=C, numbers=none,caption=Utilisation d'un pointeur de fonction]
	nomPtrFonction (liteDesArguments);
\end{lstlisting}
\subsection{Exemple}
\begin{lstlisting}[language=C, numbers=none,caption=Exemple d'utilisation d'un pointeur de fonction]
	/* Module 1 */
int fctTest(int(*f)(int), int p) {
	return f(p);
}
	/* Module 2 */
#include "module1.h"
int toto(int a) {
	return a*a;
}
int main(void) {
	int res = fctTest(&toto, 10);
}

\end{lstlisting}


