\documentclass[12pt,a4paper,openany]{article}
\usepackage{lmodern}
\usepackage{xcolor}
\input{../includesLaTeX/couleurs.tex}

\usepackage[utf8]{inputenc} \usepackage[T1]{fontenc}
\usepackage[francais]{babel}
\usepackage[top=1.7cm, bottom=1.7cm, left=1.7cm, right=1.7cm]{geometry}
\usepackage{verbatim}
\usepackage[urlbordercolor={1 1 1}, linkbordercolor={1 1 1}, linkcolor=vert1, urlcolor=bleu, colorlinks=true]{hyperref}
\usepackage{tikz} %Vectoriel
\usepackage{listings}
\usepackage{fancyhdr}
\usepackage{multido}
\usepackage{amssymb}
\usepackage{float}

\newcommand{\titre}{Dossier de spécifications des besoins logiciels}

\newcommand{\pole}{}
\newcommand{\sigle}{}

\newcommand{\semestre}{4}

\input{../includesLaTeX/listings.tex}
\input{../includesLaTeX/l2/docsProjet2.tex}
\input{../includesLaTeX/remarquesExempleAttention.tex}
\input{../includesLaTeX/polices.tex}
\makeatletter
\def\thickhrulefill{\leavevmode \leaders \hrule height 1ex \hfill \kern \z@}

\newlength{\sectiontitleindent}
\newlength{\subsectiontitleindent}
\newlength{\subsubsectiontitleindent}
\setlength{\sectiontitleindent}{-1cm}
\setlength{\subsectiontitleindent}{-.5cm}
\setlength{\subsubsectiontitleindent}{-.25cm}

\renewcommand{\section}{%
	\@startsection%
	{section}%
	{1}%
	{\sectiontitleindent}%
	{-3.5ex plus -1ex minus -.2ex}%
	{2.3ex plus.2ex}%
	{\sectionfont\Large}
}
\renewcommand{\subsection}{%
	\@startsection%
	{subsection}%
	{2}%
	{\subsectiontitleindent}%
	{-3.5ex plus -1ex minus -.2ex}%
	{2.3ex plus.2ex}%
	{\sectionfont\large}
}

\renewcommand{\subsubsection}{%
	\@startsection%
	{subsubsection}%
	{3}%
	{\subsubsectiontitleindent}%
	{-3.5ex plus -1ex minus -.2ex}%
	{2.3ex plus.2ex}%
	{\sectionfont\normalsize}
}

\makeatother

\newcommand{\lien}[1]{
 $\vartriangleright$ \url{#1}
 }

\newcommand{\pfp}{\texttt{pfp}}

\newcommand{\ifp}{\texttt{if}}
\newcommand{\elsep}{\texttt{else}}

\makeatother
\includeonly {
}
\begin{document}
	\setcounter{tocdepth}{2}
	\setcounter{secnumdepth}{3}
	\maketitle
	\tableofcontents
	\newpage
	\section{Avant-propos}
		\subsection{But du document}
	Le but de ce document est de lister toutes les fonctionnalités du futur logiciel et de son contexte d’utilisation (utilisateurs, autres composantes,
	matériel,  etc.).
		\subsection{Contexte de l'application}
		Cette application sera développé par Antoine de \bsc{Roquemaurel} et Fabrice \bsc{Valleix} dans le cadre du projet logiciel du 
		semestre 4 de la L2 Informatique de l'université Toulouse III -- Paul Sabatier.

	
	\section{Description globale} 
	Le logiciel sera un jeu de Boggle$^{\tiny\textregistered}$.

	Le jeu prend la forme d'une grille carrée, la taille étant donné par l'utilisateur. Dans chaque case de la grille est présente une lettre, au
	commencement du jeu, un compte à rebours de trois minutes est lancé, durant ces trois minutes le joueur doit chercher le plus de mots pouvant être
	formés à partir de lettre adjacentes du plateau, les mots peuvent doivent être de plus de 3 lettres et être présent dans le dictionnaire afin d'être
	acceptés.
	\subsection{Environnement}
		Le logiciel pourra être utilisé sur un ordinateur classique, sous GNU/Linux, aucun matériel externe ne sera nécessaire au bon fonctionnement du
		programme, de même le serveur d'interface graphique \texttt{X} ne sera pas utile, en effet le programme fonctionnera en mode texte ou semi graphique.

		Aucun matériel externe ne sera nécessaire.
	\subsection{Profil des utilisateurs}
		Le logiciel possédera un unique acteur.
		\begin{description}
			\item[Le joueur] Il veut faire une partie de Boggle, l'ordinateur en génére une puis le joueur cherche des mots pendant 3 minutes en
				essayant de faire le plus de points possible.
		\end{description}

	\section{Spécifications générales} % TODO
	\subsection{Description des services attendus}
		\begin{description}
			\item[Résoudre une grille par l'ordinateur]
			\item[Générer une grille]
			\item[Jouer] 
		\end{description}
	\subsection{Description générale des fonctions}
		\begin{description}
			\item[Génration d'une grille] 
			\item[Lancement d'une partie] 
			\item[Proposition d'un mot] 
			\item[Demande d'aide pour une lettre donnée]
			\item[Affichage de la solution] 
			\item[Calcul du nombre de points obtenus] 
		\end{description}
		
		% Générer nouvelle grille
		% Commencer partie (lancer timer)
		% Proposer un mot
			% Accepté ou refusé
		% Demander de l'aide pour une lettre donnée
		% Afficher solution
		% Calculer nombre de points obtenus

	\section{Exigences opérationnelles} % TODO
	\subsection{Contrainte d'exploitation}
	\subsection{Modes de fonctionnement}
	Le logiciel disposera de deux modes de fonctionnement différents :
	\begin{itemize}
		\item Le mode texte
		\item Le mode pseudo graphique
	\end{itemize}
	Ces deux modes seront choisis en fonction des arguments du programme, par défaut l'application sera lancée avec le mode pseudo graphique.
		% Deux modes de fonctionnement : Mode texte ou pseudo-graphique
	\subsection{Capacités}
		% 
	\subsection{Performances}
		% 

	\section{Scénarii d'utilisations}
		% Résolver une grille
			 % Fournir grille
			 % Si la grille est correct
				 % Afficher solution
			 % Sinon exception
		% Faire une partie (3 minutes)
			% Générer grille
			% lancer timer
			% Proposer mots
				% si mot correct
					% calculer point
				% sinon, rejeter le mot et afficher message
			% timer claque
				% Calculer nombre de point total
				% Afficher mots manquants
	\appendix
	\listoffigures
\end{document}

